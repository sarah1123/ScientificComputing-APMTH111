\documentclass[12pt,letterpaper,noanswers]{exam}
\usepackage[usenames,dvipsnames,svgnames,table]{xcolor}
\usepackage[margin=0.9in]{geometry}
\renewcommand{\familydefault}{\sfdefault}
\usepackage{multicol}
\pagestyle{head}
\header{AM 111 Class 01}{}{Math in computers, p.\thepage}
\runningheadrule
\headrule
\usepackage{siunitx}
\usepackage{enumitem}
\usepackage{graphicx} % more modern
\usepackage{amsmath} 
\usepackage{amssymb} 
\usepackage{hyperref}
%\usepackage{tcolorbox}
% % Python environment
% \lstnewenvironment{python}[1][]
% {
% \pythonstyle
% \lstset{#1}
% }
% {}

% % Python for external files
% \newcommand\pythonexternal[2][]{{
% \pythonstyle
% \lstinputlisting[#1]{#2}}}

% % Python for inline
% \newcommand\pythoninline[1]{{\pythonstyle\lstinline!#1!}}
% \newcommand{\vc}[1]{\underline{#1}}


\usepackage[most]{tcolorbox}
\usepackage{listings}

\definecolor{white}{rgb}{1,1,1}
\definecolor{mygreen}{rgb}{0,0.4,0}
\definecolor{light_gray}{rgb}{0.97,0.97,0.97}
\definecolor{mykey}{rgb}{0.117,0.403,0.713}

\tcbuselibrary{listings}
\newlength\inwd
\setlength\inwd{1.3cm}
% https://tex.stackexchange.com/questions/340700/ipython-notebook-input-and-output-cells-with-listings
\newcounter{ipythcntr}
\renewcommand{\theipythcntr}{\texttt{[\arabic{ipythcntr}]}}

\newtcblisting{pyin}[1][]{%
  sharp corners,
  enlarge left by=\inwd,
  width=\linewidth-\inwd,
  enhanced,
  boxrule=0pt,
  colback=light_gray,
  listing only,
  top=0pt,
  bottom=0pt,
  overlay={
    \node[
      anchor=north east,
      text width=\inwd,
      font=\footnotesize\ttfamily\color{mykey},
      inner ysep=2mm,
      inner xsep=0pt,
      outer sep=0pt
      ] 
      at (frame.north west)
      {\refstepcounter{ipythcntr}\label{#1}In \theipythcntr:};
  }
  listing engine=listing,
  listing options={
    aboveskip=1pt,
    belowskip=1pt,
    basicstyle=\footnotesize\ttfamily,
    language=Python,
    keywordstyle=\color{mykey},
    showstringspaces=false,
    stringstyle=\color{mygreen}
  },
}
\newtcblisting{pyprint}{
  sharp corners,
  enlarge left by=\inwd,
  width=\linewidth-\inwd,
  enhanced,
  boxrule=0pt,
  colback=white,
  listing only,
  top=0pt,
  bottom=0pt,
  overlay={
    \node[
      anchor=north east,
      text width=\inwd,
      font=\footnotesize\ttfamily\color{mykey},
      inner ysep=2mm,
      inner xsep=0pt,
      outer sep=0pt
      ] 
      at (frame.north west)
      {};
  }
  listing engine=listing,
  listing options={
      aboveskip=1pt,
      belowskip=1pt,
      basicstyle=\footnotesize\ttfamily,
      language=Python,
      keywordstyle=\color{mykey},
      showstringspaces=false,
      stringstyle=\color{mygreen}
    },
}
\newtcblisting{pyout}[1][\theipythcntr]{
  sharp corners,
  enlarge left by=\inwd,
  width=\linewidth-\inwd,
  enhanced,
  boxrule=0pt,
  colback=white,
  listing only,
  top=0pt,
  bottom=0pt,
  overlay={
    \node[
      anchor=north east,
      text width=\inwd,
      font=\footnotesize\ttfamily\color{mykey},
      inner ysep=2mm,
      inner xsep=0pt,
      outer sep=0pt
      ] 
      at (frame.north west)
      {\setcounter{ipythcntr}{\value{ipythcntr}}Out#1:};
  }
  listing engine=listing,
  listing options={
      aboveskip=1pt,
      belowskip=1pt,
      basicstyle=\footnotesize\ttfamily,
      language=Python,
      keywordstyle=\color{mykey},
      showstringspaces=false,
      stringstyle=\color{mygreen}
    },
}





%\newcommand{\note}[1]{\textcolor{red}{[#1]}} % show notes in red
 \newcommand{\note}[1]{} % don't display notes

\begin{document}
 \pdfpageheight 11in 
  \pdfpagewidth 8.5in

\noindent 

\note{calendar:
\begin{enumerate}
    \item Tu intro PS01
    \item Th binary PS01
    \item Tu condition number PS01/2
    \item Th least sq PS02
    \item Tu lin alg PS02/3
    \item Th lin alg PS03
    \item Tu least sq PS03/4
    \item Th ?? PS04
    \item Tu root finding PS04/5
    \item Th root finding PS05 (early)
    \item Tu integration PS06
    \item Th quiz
    \item Tu interpolation PS06
    \item Th interpolation PS06
    \item Tu integration PS07
    \item Th Monte Carlo PS07
    \item Tu differentiation PS08
    \item Th differentiation PS08
    \item Tu diff eq
    \item Th application of diff eq
    \item Tu ODEs
    \item Th ODEs
    \item Tu neural nets
    \item Tu neural nets
    \item Th quiz
    \item Tu presentations
\end{enumerate}}

\note{
\begin{itemize}
    \item what is a floating point system
    \item example
    \item IDing info about a floating point system
    \item 
\end{itemize}
}
\setcounter{section}{-1}
\section{Preliminaries}
\begin{itemize}
\itemsep0pt
\item There will be a skill check in class during the next class.  The problem info is below.
\item Problem set 01 will be assigned Friday and due the following Friday.
\item OH this week: Thursday from 1-2pm in Pierce 316 (Sarah's office)
\end{itemize}

\hrule
\vspace{0.2cm}


\noindent\textbf{Big picture}
\begin{itemize}
    \itemsep0pt
\item What is scientific computing (and numerical analysis)?  What makes it important?

\item How are numbers represented in computers?  What implications does this have for solving problems?
\end{itemize}

\vspace{0.2cm}
\hrule
\vspace{0.2cm}

\noindent \textbf{Skill check practice}
Write a binary number in base $10$.  

\noindent Example:

Write $(1011010.101)_2$ in base $10$.


\vspace{0.2cm}
\hrule
\vspace{0.2cm}

\noindent \textbf{Skill check solution}

Working from the radix point (decimal point) to the right to tackle the fraction: $1/2 + 1/8$.

Working from the radix point to the left to tackle the integer part: $2 + 8 + 16 + 64$.

Now some arithmetic: $0.5 + 0.125 = 0.625$.  $2+8 = 10$.  $16 + 64 = 80$.  $10 + 80 = 90$.

$90.625$ in base $10$.

\vspace{0.2cm}
\hrule
\vspace{0.2cm}

\section{Math with computers}
\subsection{Math, science, and data problems}
\begin{enumerate}
\item What have you used computers to calculate?
\end{enumerate}
\noindent \emph{Make a list for yourself, and then work with others to make a list at the boards.}
\vspace{1in}

\subsection{Challenges of doing math with computers: error}
\begin{enumerate}[resume]
\item How many numbers are there in the interval $[0,1)$?  
\end{enumerate}
\vspace{0.2in}

\noindent In a computer, we assign finite memory to store each number.  
\begin{itemize}
\itemsep0pt
    \item Only some numbers can be exactly represented: there is no continuity.  Every number has a neighborhood about it with no other numbers.
    \item Numbers for which the representation would be infinite are always shortened (ex: $\pi$, $\sqrt{2}$, $0.\bar{3}$).
\end{itemize}

\vspace{1in}

\section{Question formulation technique}
Question generation is an important skill and is not one we always approach in a formal way.

We will use a structured question formulation technique to generate questions about doing math with computers.

This technique is from The Right Question Institute, \url{rightquestion.org}.

\subsection{Technique steps}
\begin{enumerate}
    \item Produce questions
    \begin{enumerate}
        \item Ask as many questions as you can
        \item Do not stop to discuss, judge, or answer
        \item Record \textbf{exactly} as stated
        \item Change statements into questions
    \end{enumerate}
    \item Expand questions
    \begin{enumerate}
    \item Classify questions as open/closed
    \item Change from one type to the other
    \end{enumerate}
    \item Prioritize
    \item Reflect
\end{enumerate}

\subsection{Generating questions}
\subsubsection{Question focus}
\begin{pyin}
print(3/2 - 1/2 - 1)
\end{pyin}

\begin{pyprint}
0.0
\end{pyprint}

\begin{pyin}
print(7/3 - 4/3 - 1)
\end{pyin}

\begin{pyprint}
2.220446049250313e-16
\end{pyprint}

\subsubsection{Record priority questions}
\vspace{1in}

\subsubsection{Fill out reflection}
\url{PollEv.com/apmth111}

\section{Representing numbers in a computer}
\subsection{Binary (Sauer \S 0.2)}

In a base 10 number system, each digit takes on values from \{0,1,2,3,4,5,6,7,8,9\}.  In a binary number system, each binary digit (\textbf{bit}) is 0 or 1.

\vspace{0.5in}

%\href{https://watermark-silverchair-com.ezp-prod1.hul.harvard.edu/9780262334426_cab.pdf?token=AQECAHi208BE49Ooan9kkhW_Ercy7Dm3ZL_9Cf3qfKAc485ysgAAAz4wggM6BgkqhkiG9w0BBwagggMrMIIDJwIBADCCAyAGCSqGSIb3DQEHATAeBglghkgBZQMEAS4wEQQMv0aSsM8wsEnAbh4oAgEQgIIC8WLUspKb1WQLo_-Lp-zgfBYXEy_o-UeljoCmwqa6uyrVyq-VCq7G0J3ubwjlyiXo3Kt34PfZ-MweoTV4dpn9D16FH9ymoCX6MXN6mXqTwn3pPnD3Q04DDgWu7zzP_qxf5TQ7XhpW7PHkUOygwaFHCFYsCCxYhB3hYKZza2f8v_Ieh6XebCUm-S4riV2E4PXybApX6g89lliplm_BdJ54M4L6Gu4cWiuEhAX23v-DZVJNjt8CipGfgwLAp8OodgJfsWaZPqZ7QDrn2UpjxTIVFeCEYZV_NTsi73oGpnOzDjfA2gh_05bZJ-5aImkv67mL2o4zanJcg1-rmbRGHxsF68VBto_oXYGdiy6n5__F6PieIufsndzw0_sXxN2ceG-9D6TQbmBGJxTx_eeZlpaUa6f09ZYTMO0egcp63xiboRnQ4qaF1Z-QnaYBGpfxdwV8RS-iF292dsEWkpiUjLnfMbNwoqyQz4n4eJbubGHGjXnNJZrzy4rAgf8X3Btfvlo6ozXHAezyALyRG6pRfin_qpSZZs3LWZL1-Mj4CGqMRK4S5NgHlTv_yPJXbU3sAIo-pjfY5272U8lDkTPEzEptJUxos4A_5j9_ttdfLcUwP5EB6wLqL5jbFeXVNEQOA5YHgdD1OymFsFRAlFzM-2MT92JC9FznoIC3ReQ7oXJoAdTa3aK2b7sVnAaJPPhCakHdWG4rHDYfhzjaqXxYdVoZkuqyu89hMA5oF6M6cskrzyblT93gdzMLCQPYK7FbCSqRbsSOvmBKxo70SzBJ4rxlBw0SxaKwjOu7BvBda4wOcM1fcvO2OX_ngbUAWUeKhndoimOZaIFiUntb2TzewPjgWrkYtguFeTqbiHYond6Z8bM-6SNYoo5qIcREXLLdZilQupSERwK-NpuRUvjV7r1no6QmqriSAPmeq8ayRCBENdY50SlH7Vew3tBZIYnvofVz7vzn0pWnEwDptojlzbMbCNgqVVEa3Icip0f8JY1fIhDqpw}{Storing numbers in ENIAC}
\begin{enumerate}[resume]
\item In the 1940s, ENIAC was an early computer that stored decimal numbers.  A number of early computers used decimal.  Assume you have stored a single decimal digit using $5$ bits.  How many numbers can be stored using binary with five bits?
\vspace{1in}
\end{enumerate}

\subsection{Converting between bases}


\begin{enumerate}[resume]
\itemsep40pt
    \item Converting from binary to decimal 
    \begin{enumerate}
    \itemsep40pt
        \item $(11)_{10}$ (eleven in base ten) can be thought of as $1*10^1+1*10^0$.  If we are instead in base $2$, $(11)_2 = 1*2^1+1*2^0$. What does $(11)_2$ convert to in base 10?
        \item Write $(101)_2$ in base $10$.
        \item Write $(101.1)_2$ in base $10$.
    \end{enumerate}
    \item Converting integers from decimal to binary
    \begin{enumerate}
    \itemsep40pt
        \item $(16)_{10}$ = $(\_ \_ \_ \_ \_)_2$.  Find the base $2$ representation of $16$.
        \item Find the base $2$ representation for $17$.
    \end{enumerate}
\end{enumerate}

\vspace{0.5in}

\textbf{Algorithm for base 10 integer to binary:}
\begin{tcolorbox}


In base $10$ when you divide an integer by $10$, the remainder is the right most digit of the integer: $17/10 = 1\ R7$ where $7$ is the right most digit.

We can continue to divide by $10$ and read off the remainder as the next digit.  For a decimal number, we would do that until we have a single digit and a remainder

By listing the digit and then the remainders in the right order, you can reconstruct the number. \\


Something similar works in binary.

When you divide an integer by $2$ the remainder is a $0$ or a $1$:
$17/2 = 8 R1$ where $1$ will be the right most bit of the binary representation.


\end{tcolorbox}


$17/2 = 8 R1$.  $(17)_{10} = (\_ \_ \_ \_ 1)_2$.

$8/2 = 4 R0$.  $(17)_{10} = (\_ \_ \_ 0 1)_2$.

$4/2 = 2 R0$.  $(17)_{10} = (\_ \_ 0 0 1)_2$.

$2/2 = 1 R0$.  $(17)_{10} = (1 0 0 0 1)_2$.

\begin{enumerate}[resume]
    \item Use division by $2$ and keep the remainders to find $27$ in binary.
\end{enumerate}
\vspace{1.2in}

\textbf{Algorithm for base 10 fractional or decimal part to binary:}

\begin{tcolorbox}
    To convert an integer to binary, we divided by $2$ and used the remainder.  To convert a decimal or fractional part to binary, we will multiply by $2$ and read off the first digit.
\end{tcolorbox}

Find $\dfrac{1}{3}$ in binary.

$\dfrac{1}{3} = (0. \_ \_ \_ \_)_2$.

$\dfrac{1}{3}*2 = \dfrac{2}{3}$.    $\dfrac{1}{3} = (0.0 \_ \_ \_)_2$.

$2/3 * 2 = 4/3 = 1 + 1/3$.  $\dfrac{1}{3} = (0.01\_ \_)_2$. 

The fractional part is now $\dfrac{1}{3}$. 

Wait... we've already done $1/3$ so we know what happens from here.  This is a repeating binary expansion.  

$\dfrac{1}{3} = (0.\overline{01})_2$


% \begin{enumerate}[resume]
%     \item Use multiplication by $2$ and keep the integer parts to find $0.3$ in binary.
%     \item Express $27.3$ as a floating point number using double precision and rounding to nearest.
%     \begin{parts}
%     \item First, convert $27.3$ to binary.
%     \item Example notation for expressing the floating point number (have 52 bits in the box):
% \[\text{fl}(\frac{1}{3}) = +1.\ \framebox[11cm][l]{0101010101010101010101010101010101010101010101010101}\times 2^{-2}\]

% Follow this notation (and use the round to nearest rule) to write $27.3$.
%     \end{parts}
    
    
%     \item 

% Use this notation to write $27.3$ as a floating point number (with $\beta = 2$, $p = 53$)
% \end{enumerate}

% \eject

% \noindent\textbf{Some answers}
% \begin{enumerate}
%     \item a) $+9999\times 10^5$.  b) $+9.999\times 10^5$.  c) $243$.  d) $-3.400\times 10^{-3}$.  e) $+1.000\times 10^0$.  f) $1.001\times 10^0$ 
%     \item a) $1.001\times 10^0$ is the next number, so $\epsilon = 1.000\times10^{-3}$.  b) $\pi = 3.141592...$ so $3.142$ (round) or $3.141$ (truncate) c) $+1.000\times10^{-5}$ so $0.00001$. d) $0 = 0.0000$ but there has to be a nonzero number to the left, so not clear how to, no.
%     \item a) $+2.000\times10^{-5}$ b) $1.000 + 0.00001 = 1.00001$ but we stop at $4$ digits so $+1.000\times 10^0$.  c) $-1.000 + 0.00001$ is $-1.000$ so $0$.
%     \item \begin{tabular}{c|c|c}
%      & chop & nearest \\
%      \hline
%   2.394   & 2.3& 2.4\\
% 2.251 & 2.2& 2.3 \\
% 2.250 & 2.2 & 2.2 \\
% \end{tabular}
% $2.25$ was between $2.2$ and $2.3$ so look to the $3$ to decide which way to round.
% \item chop: $\vert(2.394-2.3)/2.394\vert = 0.94/2.394$.  round: $\vert(2.394-2.4)/2.394\vert = 0.006/2.394$.  chopping is worse (something like 40\% !?)
% \item a) $(11)_2 = 1+2 = 3$. b) 4+0+1 = 5.  c) $5+ 1*2^{-1} 5.5$.
% \item a) $16 = 2^4$ so $(10000)_2$. b) $17 = 16+1$ so $(10001)_2$.
% \item $27/2 = 13 R1$.  $13/2 = 6 R1$.  $6/2 = 3 R0$.  $3/2 = 1 R1$.  $(11011)_2$.  Check: $16+8+0+2+1 = 27.$
% \item $0.3*2 = 0.6$ so $(0.0...)_2$.  $0.6*2 = 1.2$ so $(0.01...)_2$.  $0.2*2 = 0.4$ so $(0.010...)_2$.  $0.4*2 = 0.8$ so $(0.0100...)_2$.  $0.8*2 =1.6$ so $(0.01001...)_2$.  We have seen $0.6$ onward before so $1001$ will repeat.  $(0.0\overline{1001})_2$.  Check: $x = (0.0\overline{1001})_2$.  $2x = (0.\overline{1001})_2$.  $2*2^4x = (1001.\overline{1001})_2$.  $(1001)_2 = 9$ so $2x + 9 = 32x$.  $30x = 9 \Rightarrow x = 3/10 = 0.3$
% \item a) $27.3 = 27+0.3 = (11011.0\overline{1001})_2$  b) $+1.1011010011001100\times 2^4$ <-- 12 bits after the radix point. $12+8*5 = 52$ so make 5 copies of the last 8:

% $+1.10110100110011001100110011001100110011001100110011001100\times 2^4$.  For rounding, the next bits are $11$ so round up

% $+1.\framebox[11.7cm][l]{10110100110011001100110011001100110011001100110011001101}\times 2^4$
% \end{enumerate}



% \subsection{Floating point (Sauer \S 0.3)}
% \begin{tcolorbox}
% \begin{itemize}
% \itemsep0pt
%     \item A \textbf{floating point number} consists of three parts: the \textbf{sign} of the number, the string of bits (\textbf{mantissa}), and an \textbf{exponent}.  \cite{sauer2018numerical}
%     \item These parts are stored together.
%     \item A floating point system can be characterized by four numbers:
% \begin{itemize}
% \itemsep0pt
%     \item $\beta$ (base)
%     \item $p$ (precision)
%     \item $[L,U]$ (range for exponents)
% \end{itemize}

% \item A floating point number is of the form \[ \pm \left(d_0 + d_1\beta^{-1} + d_2 \beta^{-2} + ... + d_{p-1}\beta^{-(p-1)}\right) \beta^E\] where $0\leq d_i\leq \beta-1$ and $L\leq E \leq U$ are integers.

% In base $10$ you would write $d_0.d_1d_2...d_{p-1} \times 10^E$.
% \item $\pm$ is the sign, $d_0d_1d_2...d_{p-1}$ is the mantissa, $E$ is the exponent, and $p$ is the precision.
% \item In a \textbf{normalized} system we require $d_0\neq 0$.
% \end{itemize}

% \end{tcolorbox}

% \noindent\textbf{Example} 

% THIS LED TO LOTS OF CONFUSION - FORGOT TO PRESENT NORMALIZED...

% Consider a normalized floating point system with $\beta = 10$, $p = 4$, $-5\leq E \leq 5$.
% \begin{itemize}
%     \item For numbers in this system, find the smallest number $\epsilon$ such that $1+\epsilon > 1$.  This number is referred to as machine epsilon, $\epsilon_{\text{mach}}$
%     \vspace{0.7in}
    
%     \item How would you represent $\pi$ in this system?  \emph{If there are multiple options, explain your choice.}
%     \vspace{1in}
% \end{itemize}


% \begin{itemize}
% \item Find the smallest number represented in the system.  This is the underflow level (UFL).
% \vspace{0.5in}

% \item Find the largest number represented in the system.  This is the overflow level (OFL).
% \vspace{0.5in}

% \item Compute $1.0\times 10^{-5}+ 1$ within this system.
% \vspace{0.5in}
% \end{itemize}

% Note: $1.0\times 10^{-5} + 1.0\times 10^{-5} = 2\times 10^{-5}$ 

% but $(\num{1.0e-5} + 1) + (\num{1.0e-5}-1) = 0$.  This is called \textbf{catastrophic cancellation}.

% \vspace{0.2cm}
% \hrule
% \vspace{0.2cm}

% \noindent\textbf{Rounding}
% \begin{tcolorbox}
% \begin{itemize}
% \itemsep0pt
%     \item Let $\text{fl}(x)$ denote the floating point approximation to $x$.
%     \item To convert from $x$ to $\text{fl}(x)$ one option is to \textbf{chop}, taking the first $p$ digits.  
    
%     Dropping the $p+1$st digit leads to a relative error of $\left\vert\dfrac{\text{fl}(x) - x}{x}\right\vert\leq \beta^{-(p-1)}$
%     \item Another option is \textbf{rounding to nearest} by taking the floating point number nearest to $x$.  
    
%     This leads to a relative error of $\left\vert\dfrac{\text{fl}(x) - x}{x}\right\vert\leq \frac{1}{2}\beta^{-(p-1)}$
    
%     Note: If $x$ is equidistant between two floating point numbers (think of $1.5$ when rounding to the nearest integer), then the value of $d_{p-1}$ is used to determine whether to round up or down.  
%     \item The maximum relative error gives the accuracy of the floating point system.  It is \[\epsilon_{\text{mach}} = \max\limits_{x\in[\text{UFL,OFL}]} \left\vert\dfrac{\text{fl}(x) - x}{x}\right\vert.\]
% \end{itemize}
% \end{tcolorbox}

% \noindent\textbf{Example}.

% Let $\beta = 10$, $p =2$.

% \begin{tabular}{c|c|c}
%      & chop & nearest \\
%      \hline
%   2.344   & & \\
% 2.351 & & \\
% 2.389 & & \\
% 2.350 & & \\
% \end{tabular}


% \vspace{0.2cm}
% \hrule
% \vspace{0.2cm}
% \noindent\textbf{Binary}

% \begin{tcolorbox}
% \begin{itemize}
% \itemsep0pt
%     \item The floating point systems used in computers typically have $\beta = 2$ (binary), so the digits of the floating point numbers, $d_i = 0$ or $1$.  We will use $b_i$ interchangeably with $d_i$ when we are working in binary.
%     \item A single binary digit is called a \textbf{bit}.
%     \item In double precision, a floating point number has $64$ bits.  \[s e_1e_2...e_{11}b_1b_2...b_{52}\] where $s$ is the sign, followed by $11$ bits for the exponent, and then the $52$ bits following the decimal point (also called the \textbf{binary point} or \textbf{radix point}).
%     \item A few exceptional numbers also need to be represented: NaN (not a number, i.e. 0/0), Inf (infinity, i.e. $1/0$), 0 (note that $d_0 = 1$ so $0$ needs a special representation)
% \end{itemize}
% \end{tcolorbox}

% \vspace{0.2cm}
% \hrule
% \vspace{0.2cm}

% % \noindent\textbf{Converting between binary and decimal}

% % One method for decimal integer to binary: 
% % \begin{itemize}
% % \itemsep0pt
% %     \item To read off the digits of $548$, divide by $10$ (shift the decimal place to the left) and keep the remainder.  
    
% %     $548/10 = 54.8 = 54$ R$8$ 
    
% %     $54/10 = 5.4 = 5$ R$4$
    
% %     Reconstruct the number: $548$.
% %     \item To find the bits, divide by $2$ (shift the binary place to the left) and keep the remainder.
    
% %     $548/2 = 274$ R$0$ 
    
% %     $274/2 = 137$ R$0$
    
% %     $137/2 = 68$ R$1$
    
% %     $68/2 = 34$ R$0$
    
% %     $34/2 = 17$ R$0$
    
% %     $17/2 = 8$ R$1$
    
% %     $8/2 = 4$ R$0$
    
% %     $4/2 = 2$ R$0$
    
% %     $2/2 = 1$ R$0$
    
% %     Now reconstruct: $1000100100 = 2^2+2^5+2^{9} = 4 + 32 + 512 = 548$.
    
% % \end{itemize}

% % \noindent\textbf{Examples}


% % Convert $3$ to binary.
% % \vspace{0.5in}

% % Convert $700$ to binary.
% % % \vspace{1in}



% \bibliographystyle{plain}
% \bibliography{AM111references.bib}


\end{document}