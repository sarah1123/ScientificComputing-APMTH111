\documentclass[12pt,letterpaper,noanswers]{exam}
\usepackage[usenames,dvipsnames,svgnames,table]{xcolor}
\usepackage[margin=0.9in]{geometry}
\renewcommand{\familydefault}{\sfdefault}
\usepackage{multicol}
\pagestyle{head}
\header{AM 111 Class 02}{}{Floating point, p.\thepage}
\runningheadrule
\headrule
\usepackage{siunitx}
\usepackage{enumitem}
\usepackage{graphicx} % more modern
\usepackage{amsmath} 
\usepackage{amssymb} 
\usepackage{hyperref}
%\usepackage{tcolorbox}
% % Python environment
% \lstnewenvironment{python}[1][]
% {
% \pythonstyle
% \lstset{#1}
% }
% {}

% % Python for external files
% \newcommand\pythonexternal[2][]{{
% \pythonstyle
% \lstinputlisting[#1]{#2}}}

% % Python for inline
% \newcommand\pythoninline[1]{{\pythonstyle\lstinline!#1!}}
% \newcommand{\vc}[1]{\underline{#1}}


\usepackage[most]{tcolorbox}
\usepackage{listings}

\definecolor{white}{rgb}{1,1,1}
\definecolor{mygreen}{rgb}{0,0.4,0}
\definecolor{light_gray}{rgb}{0.97,0.97,0.97}
\definecolor{mykey}{rgb}{0.117,0.403,0.713}

\tcbuselibrary{listings}
\newlength\inwd
\setlength\inwd{1.3cm}
% https://tex.stackexchange.com/questions/340700/ipython-notebook-input-and-output-cells-with-listings
\newcounter{ipythcntr}
\renewcommand{\theipythcntr}{\texttt{[\arabic{ipythcntr}]}}

\newtcblisting{pyin}[1][]{%
  sharp corners,
  enlarge left by=\inwd,
  width=\linewidth-\inwd,
  enhanced,
  boxrule=0pt,
  colback=light_gray,
  listing only,
  top=0pt,
  bottom=0pt,
  overlay={
    \node[
      anchor=north east,
      text width=\inwd,
      font=\footnotesize\ttfamily\color{mykey},
      inner ysep=2mm,
      inner xsep=0pt,
      outer sep=0pt
      ] 
      at (frame.north west)
      {\refstepcounter{ipythcntr}\label{#1}In \theipythcntr:};
  }
  listing engine=listing,
  listing options={
    aboveskip=1pt,
    belowskip=1pt,
    basicstyle=\footnotesize\ttfamily,
    language=Python,
    keywordstyle=\color{mykey},
    showstringspaces=false,
    stringstyle=\color{mygreen}
  },
}
\newtcblisting{pyprint}{
  sharp corners,
  enlarge left by=\inwd,
  width=\linewidth-\inwd,
  enhanced,
  boxrule=0pt,
  colback=white,
  listing only,
  top=0pt,
  bottom=0pt,
  overlay={
    \node[
      anchor=north east,
      text width=\inwd,
      font=\footnotesize\ttfamily\color{mykey},
      inner ysep=2mm,
      inner xsep=0pt,
      outer sep=0pt
      ] 
      at (frame.north west)
      {};
  }
  listing engine=listing,
  listing options={
      aboveskip=1pt,
      belowskip=1pt,
      basicstyle=\footnotesize\ttfamily,
      language=Python,
      keywordstyle=\color{mykey},
      showstringspaces=false,
      stringstyle=\color{mygreen}
    },
}
\newtcblisting{pyout}[1][\theipythcntr]{
  sharp corners,
  enlarge left by=\inwd,
  width=\linewidth-\inwd,
  enhanced,
  boxrule=0pt,
  colback=white,
  listing only,
  top=0pt,
  bottom=0pt,
  overlay={
    \node[
      anchor=north east,
      text width=\inwd,
      font=\footnotesize\ttfamily\color{mykey},
      inner ysep=2mm,
      inner xsep=0pt,
      outer sep=0pt
      ] 
      at (frame.north west)
      {\setcounter{ipythcntr}{\value{ipythcntr}}Out#1:};
  }
  listing engine=listing,
  listing options={
      aboveskip=1pt,
      belowskip=1pt,
      basicstyle=\footnotesize\ttfamily,
      language=Python,
      keywordstyle=\color{mykey},
      showstringspaces=false,
      stringstyle=\color{mygreen}
    },
}





%\newcommand{\note}[1]{\textcolor{red}{[#1]}} % show notes in red
 \newcommand{\note}[1]{} % don't display notes

\begin{document}
 \pdfpageheight 11in 
  \pdfpagewidth 8.5in

\noindent 

\note{calendar:
\begin{enumerate}
    \item Tu intro PS01
    \item Th binary PS01
    \item Tu condition number PS01/2
    \item Th least sq PS02
    \item Tu lin alg PS02/3
    \item Th lin alg PS03
    \item Tu least sq PS03/4
    \item Th ?? PS04
    \item Tu root finding PS04/5
    \item Th root finding PS05 (early)
    \item Tu integration PS06
    \item Th quiz
    \item Tu interpolation PS06
    \item Th interpolation PS06
    \item Tu integration PS07
    \item Th Monte Carlo PS07
    \item Tu differentiation PS08
    \item Th differentiation PS08
    \item Tu diff eq
    \item Th application of diff eq
    \item Tu ODEs
    \item Th ODEs
    \item Tu neural nets
    \item Tu neural nets
    \item Th quiz
    \item Tu presentations
\end{enumerate}}

\note{
\begin{itemize}
    \item what is a floating point system
    \item example
    \item IDing info about a floating point system
    \item 
\end{itemize}
}
\setcounter{section}{-1}
\section{Preliminaries}
\begin{itemize}
\itemsep0pt
\item There will be a skill check in class during the next class.  The problem info is below.
\item Problem set 01 will be assigned Friday and due the following Friday.
\item OH this week: Thursday from 1-2pm in Pierce 316 (Sarah's office)
\end{itemize}

\hrule
\vspace{0.2cm}


\noindent\textbf{Big picture}
\begin{itemize}
    \itemsep0pt
\item What is scientific computing (and numerical analysis)?  What makes it important?

\item How are numbers represented in computers?  What implications does this have for solving problems?
\end{itemize}

\vspace{0.2cm}
\hrule
\vspace{0.2cm}

\noindent \textbf{Skill check practice}
Given a binary number, round it to nearest, keeping 4 bits after the radix point.  The skill check will avoid the special case where the truncated part is exactly $0.00001$.

\noindent Example:

Round $(+1.00111001)_2$ to nearest, keeping 4 bits after the radix point. 


\vspace{0.2cm}
\hrule
\vspace{0.2cm}

\noindent \textbf{Skill check solution}

We check the 5th bit after the radix point to decide whether to round down or up.  Bit 5 is a $1$ so we round up.  

To round up, add $0.0001$ to the chopped value.

$1 + 1 = 10$ in binary, and $11 + 1 = 100$, so 
\begin{align*}
    &+1.0011 \\
    &+0.0001 \\
    \cline{1-2}
     &+1.0100
\end{align*}
\section{Representing numbers in a computer}

\subsection{Converting to binary, Sauer \S 0.2}
\begin{enumerate}[resume]
\itemsep60pt
    \item Use division by $2$ and tracking the remainders to find $27$ in binary.
    \item Use multiplication by $2$ and tracking the integer parts to find $0.3$ in binary.
    \item Convert $27.3$ to binary.

    \end{enumerate}

\subsection{Floating point, Sauer \S 0.3}
\begin{enumerate}[resume]
\itemsep50pt
\item Write the following numbers in a normalized way.
\begin{itemize}
\item $(0.05618)_{10}$
\item $(1001.01)_2$
\item $(0.0001101)_2$
\item $(9241.2)_{10}$
\end{itemize}

\item Is it always possible to write a number in a normalized way?  For what number(s) does it fail (if any)?
\end{enumerate}
\vspace{1in}

Consider a normalized floating point system in base $10$ with precision three digits are kept after the decimal point, and the exponent in the range $-5\leq p \leq 5$

\begin{enumerate}[resume]
\itemsep40pt
\item How would you represent $\pi$ in this system?  \emph{If there are multiple options, explain your choice.}

\item Find the smallest number represented in the system.  This is the underflow level (UFL).

\item Find the largest number represented in the system.  This is the overflow level (OFL).

\item Find an $\epsilon$ small enough that $1+\epsilon = 1$ in this floating point system.

\item Find the smallest $\epsilon$ so that $1+\epsilon > 1$ in this floating point system.

This number is referred to as machine epsilon, $\epsilon_{\text{mach}}$

\item In this system $1.0\times 10^{-5} + 1.0\times 10^{-5} = 2\times 10^{-5}$ 

What about $(\num{1.0e-5} + 1) + (\num{1.0e-5}-1)$?  \emph{Do the calculations within the parentheses first.}

This is called \textbf{catastrophic cancellation}.
\end{enumerate}

\vspace{1in}

\subsection{Shortening the numbers: rounding vs chopping}

\begin{enumerate}[resume]
\item Use base $10$ and keep $1$ digit after the decimal place.

\begin{tabular}{c|c|c}
     & chop & nearest \\
     \hline
  2.344   & & \\
2.351 & & \\
2.389 & & \\
2.35 & & \\
\end{tabular}
\end{enumerate}

\subsection{Floating point in the computer}

\begin{enumerate}[resume]
\item Express $27.3$ as a floating point number using double precision and rounding to nearest.
    
     Example notation for expressing the floating point number (have 52 bits in the box):
\[\text{fl}(\frac{1}{3}) = +1.\ \framebox[11cm][l]{0101010101010101010101010101010101010101010101010101}\times 2^{-2}\]

Follow this notation (and use the round to nearest rule) to write $27.3$ as a floating point number (with base $2$, $N = 52$).

Include the sign, the 52 bits after the radix point, and the exponent.

\end{enumerate}
\end{document}