\documentclass[12pt,letterpaper,noanswers]{exam}
\usepackage[usenames,dvipsnames,svgnames,table]{xcolor}
\usepackage[margin=0.9in]{geometry}
\renewcommand{\familydefault}{\sfdefault}
\usepackage{multicol}
\pagestyle{head}
\header{AM 111 Class 07}{}{Solving equations, p.\thepage}
\runningheadrule
\headrule
\usepackage{siunitx}
\usepackage{graphicx} % more modern
\usepackage{amsmath} 
\usepackage{amssymb} 
\usepackage{hyperref}
\usepackage{tcolorbox}
\usepackage{enumitem}
\def\mbf{\mathbf}
\newcommand{\vc}[1]{\boldsymbol{#1}}
\DeclareMathOperator*{\argmin}{arg\,min} % thin space, limits underneath in displays


\begin{document}
 \pdfpageheight 11in 
  \pdfpagewidth 8.5in

\noindent 

\section*{Preliminaries}

\begin{itemize}
\itemsep0pt
\item Problem set 03 is due on Friday at noon.
\item Problem set 03 includes some ``time permitting'' problems.  If your total time spent on the problem set reaches 8 hours in the week then you are encouraged to skip problems.  If you are not in that situation, you are expected to complete the problems.
\item There will be a skill check in class during Class 08.  The problem info is below.
\item Find all OH on Canvas.
\end{itemize}



\noindent\textbf{Big picture}

Today: Algorithms for approximating solutions to nonlinear equations.

\vspace{0.2cm}
\hrule
\vspace{0.2cm}

\noindent \textbf{Skill check practice}
\begin{questions}
\item Let $f(x) = x^2 -x -1$ and $x_0 = 0$.  Apply one step of Newton's method.
\end{questions}


\vspace{0.2cm}
\hrule
\vspace{0.2cm}

\noindent \textbf{Skill check solution}
\begin{questions}
\item $x_1 = x_0 - f(x_0)/f'(x_0)$.  

$f'(x) = 2x - 1$.

$f(0) = -1$, $f'(0) = -1$.  $x_1 = 0 - -1/(-1) = -1$

\end{questions}
\vspace{0.2cm}
\hrule
\vspace{0.2cm}
\noindent\textbf{Teams}
\begin{multicols}{2}
1. Zhecheng, Kinara, Emma

2. Ada, Trey, Paul

3. Sara, Davin, Isha

4. Kyumin, Brooke
\end{multicols}

\section*{Root finding}
\subsection*{Introduction}


\subsection*{The bisection method}

\begin{enumerate}
\itemsep50pt
\item Find the roots of $x^2-x-6$ exactly (i.e. using tools you already know for solving quadratic equations / factoring etc)
\item Use the bisection method to produce successive approximations, $x_0, x_1, ...$ to a root of $x^2-x-6$ in $[0,8]$.

\end{enumerate}
\vspace{1in}

\begin{enumerate}[resume]
    \item How many steps would be needed to find the root of $x^2-x-6$ in the interval $[0,10]$ to guarantee accuracy to $3$ decimal places?  
    


\emph{Consider a solution correct to within $q$ decimal places if the error $\vert x_i - x^*\vert < 0.5*10^{-q}$}
\vspace{1in}


    \item Let $e_i = x_i - x^*$.  Assuming $\vert e_i \vert \approx\dfrac{b-a}{2^{i+1}}$, find an approximation for $\dfrac{ \vert e_{i+1} \vert}{ \vert e_i \vert}.$
        \vspace{1in}
        
    \item When $\lim\limits_{i\rightarrow\infty} \dfrac{ \vert e_{i+1} \vert}{ \vert e_i \vert} = C$ with $0<C<1$, the convergence is said to be linear (order 1) with rate $C$.  
    
    \emph{The error is going down by a constant factor at each step.}
    
    When $C = 0$ it is superlinear (faster than linear) and when $C = 1$ it is sublinear (slower than linear).  

Is the order of convergence linear for the bisection method?  If so, what is the convergence rate?


\vspace{0.5in}

\end{enumerate}


\subsection*{Important considerations}

\begin{enumerate}[resume]
\itemsep0pt
    \item Is the bisection method guaranteed to converge?  
\end{enumerate}
\vspace{0.5in}

\subsection*{Iterated methods}


\begin{enumerate}[resume]
\itemsep0pt
    \item Let $g(x) = x - \frac{1}{10}(x^2-x-6)$.
    \begin{parts}
    \item Show that $g(x) = x$ at the root of $f(x) = x^2-x-6$.
    \vspace{1in}
    
    \item Choose a starting point of $x_0 = 0$ or $x_0 = 8$.  Use a calculator, and iterate the map to produce the sequence $x_0, x_1, ...$
    \vspace{1in}
    
    \item Use a Taylor expansion to first order, and simplify, to provide an approximation for $\dfrac{\vert \Delta x_{i+1}\vert}{\vert \Delta x_{i}\vert}$, where $x_{i+1} = x^* + \Delta x_{i+1}$.  Use the $g(x)$ above and let $x^* =3$.
    
    
    \[x_{i+1} = x^* + \Delta x_{i+1}= g(x_{i}) \approx \text{Taylor expansion of } g(x^*+\Delta x_{i}) \text{ about }x^*\]
\vspace{1in}

\part Let $g(x) = x - a(x^2-x-6)$.  Assuming $X^* = 3$, for what range of $a$ is $\dfrac{\vert \Delta x_{i+1}\vert}{\vert \Delta x_{i}\vert} < 1$?  If possible, find a value of $a$ so that it is $0$.
\vspace{0.5in}

    \end{parts}
\end{enumerate}
\subsection*{Newton's method}

\begin{enumerate}[resume]
    \item (Sauer \S1.4 1a) Apply two steps of Newton's method with initial guess $x_0 = 0$ for $f(x) = x^3+x-2$
    \vspace{1in}
    
    \item (Heath \S 5.4) Match the error sequence with convergence information.
    
    \begin{enumerate}
    \item quadratic convergence
    \item linear convergence with $C = 10^{-1}$
    \item linear convergence with $C = 10^{-2}$
    \item superlinear ($<$ quadratic)
    \end{enumerate}
    
    \begin{enumerate}
    \item[(i)] $10^{-2},10^{-4},10^{-6},10^{-8},...$
    \item[(ii)] $10^{-2},10^{-4},10^{-8},10^{-16},...$
    \item[(iii)] $10^{-2},10^{-4},10^{-7},10^{-12},...$
    \item[(iv)] $10^{-2},10^{-3},10^{-4},10^{-5},...$
    \end{enumerate}
\end{enumerate}

\subsection*{Secant method}


\subsection*{Summary}
\begin{tabular}{p{0.3\linewidth} p{0.3\linewidth} p{0.3\linewidth}}
bisection & Newton's & secant \\
\hline
always converges & may not converge & may not converge\\
sign of $f$  & needs $f$ and $f'$ & just needs $f$ \\
$f$ continuous & $f$ differentiable & $f$ continuous\\
linear convergence & quadratic convergence & superlinear ($<$ quadratic) \\
needs initial interval & one initial value& two initial values\\
no higher-dim'l version & higher dim'l version & \\
\end{tabular}

\begin{enumerate}[resume=classQ]
\item (Health \S5 5.24)

List one advantage and one disadvantage of the secant method compared with Newton's method for solving a nonlinear equation in one dimension.
\vspace{0.5in}

\item (Heath \S5 5.32) 

Think about how you might combine the idea of  a bracket around the solution from the bisection method with the secant method so that the method will still converge even if it starts far from a root.

\emph{This is called a safeguarded method}

\vspace{0.5in}
\end{enumerate}
\end{document}