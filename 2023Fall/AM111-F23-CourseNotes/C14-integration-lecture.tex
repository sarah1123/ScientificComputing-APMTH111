\documentclass[12pt,letterpaper,noanswers]{exam}
\usepackage[usenames,dvipsnames,svgnames,table]{xcolor}
\usepackage[margin=0.9in]{geometry}
\renewcommand{\familydefault}{\sfdefault}
\usepackage{multicol}
\pagestyle{head}
\header{AM 111 Class 14}{}{Approximating integrals, p.\thepage}
\runningheadrule
\headrule
\usepackage{siunitx}
\usepackage{graphicx} % more modern
\usepackage{amsmath} 
\usepackage{amssymb} 
\usepackage{hyperref}
\usepackage{tcolorbox}
\usepackage{listings}
%\usepackage[numbered,autolinebreaks,useliterate]{mcode}

\usepackage{enumitem}
\def\mbf{\mathbf}
\newcommand{\vc}[1]{\boldsymbol{#1}}
\def\dsst{\displaystyle}
\DeclareMathOperator*{\argmin}{arg\,min} % thin space, limits underneath in displays


\begin{document}
 \pdfpageheight 11in 
  \pdfpagewidth 8.5in

\noindent 

\section*{Preliminaries}

\begin{itemize}
\itemsep0pt
\item There is a problem set due Friday.
\item There is a skill check in the next class.
\end{itemize}


\noindent\textbf{Big picture}

Today: Approximating $\int_{a}^{b}f(x)dx$.

\vspace{0.2cm}
\hrule
\vspace{0.2cm}

\noindent \textbf{Skill check practice}

 A quadrature rule can be written in the form $\sum\limits_{k=1}^n a_k f(x_k)$.

For the composite midpoint rule on two panels, given by $(b-a)\frac{1}{2}\left(f(a + \frac{b-a}{4})+ f(a + 3\frac{b-a}{4})\right)$, identify $n$ and $a_k, x_k$ for $k=1,...,n$




\vspace{0.2cm}
\hrule
\vspace{0.2cm}

\noindent \textbf{Skill check solution}

There are two points in the calculation so $n = 2$.  For each point, identify $x_k$, the node, and $a_k$, the weight.

\begin{tabular}{| l|  c|}
\hline
$x_k$ & $a_k$ \\
\hline
$x_1 = a+(b-a)/4$ & $a_1 = \frac{1}{2}(b-a)$ \\
$x_2 = a+3(b-a)/4$ & $a_2 = \frac{1}{2}(b-a)$ \\
\hline
\end{tabular}


\vspace{0.2cm}
\hrule
\vspace{0.2cm}


\begin{tcolorbox}
\noindent\textbf{Simpson's rule}

Given a function $f(x) \in C^4[a,b]$, let $h = \dfrac{b-a}{2}$.
\[
          \int_{a}^{b} f(x) \, dx = \frac{h}{3}\left[ f(a)+4f\left(\dfrac{a+b}{2}\right) + f(b) \right]
          - \frac{h^5f^{(iv)}(c)}{90}.\]
\end{tcolorbox}



\begin{enumerate}[resume=classQ]
    \item Show that the degree of precision for Simpson's rule is $3$.
    
    To do this, we need to show that $\int_a^b 1\ dx$, $\int_a^b x\ dx$, $\int_a^b x^2\ dx$, $\int_a^b x^3\ dx$ are all integrated exactly by Simpson's rule.

$\int_a^b 1\ dx$, $\int_a^b x\ dx$, $\int_a^b x^2\ dx$ are integrated exactly by construction of the rule, so what is left to show is that $\int_a^b x^3\ dx$ is integrated exactly.
    \begin{parts}
    \item One way to do this is to shift your integral.  Find $k$ such that $\int_a^b x^3 dx = \int_{-h}^h (x+k)^3 dx$.
    \vspace{1cm}
    
    \item Expanding, $(x+k)^3 = x^3 + q(x)$ where $q(x)$ is a quadratic function.  
    
    $\int_a^b x^3 dx = \int_{-h}^h (x+k)^3 dx = \int_{-h}^h x^3 dx  + \int_{-h}^h q(x) dx$
    
    Provide an argument for why Simpson's rule will be exact for $\int_{-h}^h q(x)dx$.

    Check Simpson's rule on $\int_{-h}^h x^3 dx$ and show that Simpson's rule integrates this exactly, as well.

    We can conclude that $\int_a^b x^3\ dx$ is integrated exactly by Simpson's rule.
    \vspace{1.5in}
    
    
    
    \end{parts}
    
\end{enumerate}

\subsection*{Composite rules}
\begin{tcolorbox}
(from Sauer \S 5.2.3)
\begin{itemize}
\itemsep0pt
    \item For a composite rule, approximate $\displaystyle\int_a^b f(x)dx$ by breaking $[a,b]$ into adjacent subintervals, or \textbf{panels}.
    \item The \textbf{composite Trapezoid rule} is the sum of the Trapezoid rule applied to each panel.
    \item For a function $f\in C^2[a,b]$, choose an evenly spaced grid: $a = x_0<x_1<...<x_{m-1}<x_m = b$ with $h = x_k-x_{k-1} = (b-a)/m$.  \[\displaystyle\int_a^b f(x)dx = \frac{h}{2}\left(y_0 + 2\sum\limits_{k=2}^{m-1}y_k + y_m \right) - (b-a)\frac{h^2}{12}f''(c),\] where $c$ is between $a$ and $b$.
\end{itemize}
\end{tcolorbox}
\begin{enumerate}[resume=classQ]
    \item In a single panel, the error from the Trapezoid rule is given by $-\frac{1}{12}h^3f''(c_k)$ with $x_{k-1}<c_k<x_k$.
    \begin{parts}
    \item Find an expression for the sum of the error over all $m$ panels (use summation notation).
    
\vspace{1in}
\item The generalized intermediate value theorem says that for $f$ continuous on $[a,b]$, $c_k$ points in $[a,b]$, and weights $w_1, ..., w_m>0$, there exists a number $c$ between $a$ and $b$ such that $w_1f(c_1)+...+w_mf(c_m) = (w_1+...+w_m)f(c)$.  Use this theorem to write your expression as $f''(c)$ times a constant.  Write the constant in terms of $m$ and $h$.
\vspace{1in}
\item Is the error you found equal to $- (b-a)\frac{h^2}{12}f''(c)$?
\vspace{1cm}
    \end{parts}
\end{enumerate}

The error in the Trapezoid rule is proportional to $f''$ and to $h^3$.

The error in the composite Trapezoid rule is proportional to $f''$ and to $h^2$.

See the reading for details on composite Simpson's rule.

% \begin{enumerate}[resume=classQ]
% \item For composite Simpson's rule, let each panel be length $2h$.  Use the grid $a = x_0 < x_1 < ... < x_{2m-1}<x_{2m}$.
% \begin{parts}
% \item For $a = x_0, b = x_2$, Simpson's rule can be written $\int_a^b f(x)dx \approx \frac{h}{3}(y_0 + 4y_1 + y_2)$.

% The composite rule is of the form \[\frac{h}{3}\left[y_0 + A \sum\limits_{k=1}^m y_{2k-1} + B\sum\limits_{k=1}^{m-1}y_{2k} + y_{2m}\right].\]

% Find $A$ and $B$.
% \end{parts}
% \end{enumerate}

\subsection*{Open vs closed formulas}
\begin{tcolorbox}
\textbf{Closed} Newton-Cotes formulas use $f(a)$ and $f(b)$ in the approximation.

\textbf{Open} Newton-Cotes formulas avoid values from the ends of the intervals.
\end{tcolorbox}
\begin{enumerate}[resume=classQ]
    \item To approximate $\displaystyle\int_a^b f(x)dx$, let $x_0 = a, x_1 = b$, $h = b-a$.  Set $w = x_0 + h/2$ (the midpoint of the interval).
    \begin{parts}
    \item Find the degree $1$ Taylor expansion of $f(x)$ about the midpoint $w$.  Do this exactly, so include the remainder term (it is $\frac{1}{2}(x-w)^2f''(c)$.
    \vspace{1in}
    \item Integrate both sides from $a$ to $b$.
    
    Note that $f'(w)$ is a constant.  Use the mean value theorem for integrals to remove $f''(c)$ from the integral.
    \vspace{1in}
    
    \end{parts}
This generates the \textbf{midpoint rule} for approximating an integral.
\end{enumerate}
\begin{tcolorbox}
The \textbf{composite midpoint rule} is given by \[\int_a^bf(x)dx = h\sum\limits_{i=1}^m f(w_i) + \frac{1}{24}(b-a)h^2 f''(c)\] where $h = (b-a)/m$ and $a<c<b$.  The $w_i$ are midpoints of the $m$ equal subintervals of $[a,b]$.
\end{tcolorbox}
% \fbox{
%     \begin{minipage}{15cm}
%       \subsection*{Midpoint Rule}
%       Given a function $f(x) \in C^2(x_0,x_1)$ and points $x_0<x_1$, $x_1-x_0 = h$,
%             \begin{equation}
%           \int_{x_0}^{x_1} f(x) \, dx = h(f(x_0+ h/2)) + \frac{h^3}{24} f''(c)
%         \end{equation}
%     \end{minipage}}

% \vspace{10cm}


%     \fbox{
%     \begin{minipage}{15cm}
%       \subsection*{Composite Midpoint Rule}
%       Given a function $f(x) \in C^4(a,b)$ and equally spaced
%       points $a=x_0<x_1<\cdots<x_{2m}=b$, $x_{i+1}-x_i = h$,
%       \begin{equation}
%         \int_a^b f(x) \, dx = h \sum_{i=1}^m f\left(\frac{x_{i-1} + x_i}{2}\right) + \frac{(b-a)h^2f''(\xi)}{24}.
%       \end{equation}
%     \end{minipage}}

% \pagebreak

% \subsection*{Adaptive Quadrature}

% \pagebreak

% \been

% \item Read the adaptive quadrature algorithm in the book (\S5.4) and implement it to estimate:

% \beit
% \item $\int_{-1}^1   2\sqrt{1-x^2} dx$
% \item $\int_{-1}^1   1+\sin(e^{3x}) dx$
% \enit

% \enen



%
%\subsection*{Romberg Integration}
%
%\been
%
%\item[4.] Romberg integration uses the composite trapezoid rule with successively
%smaller stepsizes $h$.
%
%\been
%\item {\bf Romberg Integration (First Column)}
%
%Define stepsizes $h_j$ such that each successive stepsize is half the
%previous stepsize:
%\begin{align*}
%h_1 &= b-a \\
%h_2 &= (b-a)/2 \\ 
%h_3 &= (b-a)/4 \\
%& \vdots \\
%h_j&=\frac{1}{2^{j-1}}(b-a).
%\end{align*}
%
%The first column of Romberg integration, $R_{j1}$, is the composite
%trapezoid rule applied to the function on $[a,b]$ with stepsize $h_j$.  Write down the formula for the first few elements of the first column:
%
%\vfill
%
%\begin{align*}
%R_{11} &= \blank{3in} \\
%& \\
%R_{21} &= \blank{3in} \\
%& \\
%R_{31} &= \blank{3in}
%\end{align*}

%\pagebreak
%
%The second column applies Richardson extrapolation to the 1st column
%$R_{j1}$.  Write down the results for the second column.  What is $R_{22}$?
%
%\vfill
%
%\begin{align*}
%R_{22} &= \frac{2^2 R_{21} - R_{11}}{3} = \blank{3in} \\
%& \\
%R_{32} &= \frac{2^2 R_{31} - R_{21}}{3} = \blank{3in}
%\end{align*}
%
%\item Calculate $R_{11}$, $R_{21}$, and $R_{22}$ for $\dsst{\int_0^4 x^2 \, dx}$.
%
%\vfill
%
%\pagebreak
%
%\item The third column applies Richardson extrapolation to the second column.
%Because the second column is composite Simpson's rule, it is 4th order in $h$:
%\begin{align*}
%R_{33} &= \frac{4^2 R_{32} - R_{22}}{4^2-1} \\
%R_{43} &= \frac{4^2 R_{42} - R_{32}}{4^2-1} \\
%& \vdots \\
%R_{j3} &= \frac{4^2 R_{j2} - R_{j-1,2}}{4^2-1}
%\end{align*}
%
%\item This pattern perpetuates
%$$
%R_{jk} = \frac{4^{k-1} R_{j,k-1} - R_{j-1,k-1}}{4^{k-1}-1} 
%$$
%
%\enen
%
%\item[5.] Computer Examples
%\been
%
%\item Estimate $\dsst{\int_{-1}^1 2\sqrt{1-x^2} \, dx}$.
%
%\vspace{1cm}
%
%\item Estimate $\dsst{\int_{-1}^1 1 + \sin{e^{3x}} \, dx}$.
%
%\vspace{1cm}
%
%\enen
%
%\enen



\subsection*{Gaussian quadrature (following Sauer \S5.5)}

\begin{tcolorbox}

\begin{itemize}
\itemsep0pt
    \item A Newton-Cotes method using a quadrature of degree $n$ has degree of precision $n$ for $n$ odd and $n+1$ for $n$ even.
    \item The Trapezoid rule ($n = 1$, two points) has degree of precision $1$.
    \item Simpson's rule ($n=2$, three points) has degree of precision $3$.
    \item These methods use $n+1$ function evaluations.
\end{itemize}
\textbf{Gaussian quadrature} has degree of precision $2n-1$ when $n$ points are used.
\end{tcolorbox}
\begin{enumerate}[resume=classQ]
    \item Consider $I = \int_a^b f(x)dx$.  Let $g(x) = f(x+h+a)$, $h = (b-a)/2$.  $I = \int_{-h}^h g(x) dx$.
    
        $\int_{-h}^h g(x) dx \approx c_1 g(x_1) + c_2 g(x_2)$.  We would like to $c_1, c_2, x_1, x_2$ to maximize the degree of precision of this method.  
        
        With four unknowns that we can set, the method should be exactly correct for $g(x) = 1, x, x^2, x^3$ (degree of precision of $3$).
    \begin{parts}
\item Assuming a degree of precision of $3$, find four equations that are satisfied by $c_1, c_2, x_1, x_2$.
\vspace{1.5in}
\item This is a system of nonlinear equations.  Make an effort to find a solution to the system (you might use that $c_1x_1 = -c_2x_2$).  \emph{It is typically difficult to find a solution to a nonlinear system}.

\vfill

\item If you were to use $n$ points instead of $2$ points, how many unknowns would you have?  What degree of precision is possible (assuming the system you generate has a solution)?
\end{parts}
\end{enumerate}
\eject

\begin{tcolorbox}
\begin{itemize}
\itemsep0pt
    \item For an $n$ point Gaussian quadrature method, choosing the right $x_k$ is a little complicated.
    \item Once the $x_k$ are chosen, use Lagrange interpolation to find the quadrature weights.  %set up Lagrange basis functions $\displaystyle L_k(x) = \prod\limits_{j=1, j\neq k}^n \dfrac{x-x_j}{x_k-x_j}$
\end{itemize}

\end{tcolorbox}

\begin{enumerate}[resume=classQ]
\item You are using an $n$ point quadrature method.  Assume you have been told to use points $x_1, x_2, ..., x_n$ (where the $x_k$ values are given to you) to create an interpolating polynomial.

Let $L_k(x)$ be the basis functions for Lagrange interpolation.  

Write an expression to find $c_k$ where $\displaystyle\int_{-h}^h g(x) dx \approx \sum\limits_{k=1}^n c_k f(x_k)$.
\vspace{1in}

\end{enumerate}
\begin{tcolorbox}
Let $\{p_0(x),p_1(x),...,p_k(x),...,p_n(x)\}$ be a set of orthogonal polynomials on the interval $[-h,h]$ with $p_k$ of degree $k$.
\begin{itemize}
\itemsep0pt
    \item Set the $x_i$ to be the roots of $p_n(x)$. Then for $j\leq 2n-1$, $\displaystyle \int_{-h}^h x^j dx = \sum\limits_{i=1}^n c_i x_i^j$, i.e. the quadrature has a degree of precision of $2n-1$.
    \item The quadrature created using Lagrange interpolation at these roots $x_i$ is the \textbf{Gaussian quadrature}.
\end{itemize}


\end{tcolorbox}
\begin{enumerate}[resume=classQ]
\item %Let $P(x)$ be a polynomial of degree at most $2n-1$.

(What are orthogonal polynomials?)

Let $\{p_0(x),p_1(x),...,p_k(x),...,p_n(x)\}$ be a set of orthogonal polynomials on the interval $[-h,h]$ with $p_k$ of degree $k$.
\begin{parts}
\item Two polynomials, $f(x)$ and $g(x)$, are orthogonal on an interval $[-h,h]$ if 

$\displaystyle\int_{-h}^h f(x)g(x)dx = 0$.  Show that $p_0(x) = 1$ and $p_1(x) = x$ are orthogonal on $[-1,1]$.
\vspace{1in}
\item Let $p_2(x) = x^2 + c$.  Find $c$ so that $p_2(x)$ is orthogonal to $p_1(x)$ and to $p_0(x)$.
\vspace{1in}

\end{parts}
\end{enumerate}
\begin{tcolorbox}
Theorems (see Sauer \S 5.4)

\begin{itemize}
\itemsep0pt
    \item Let $\{p_0(x),p_1(x),...,p_k(x),...,p_n(x)\}$ be a set of orthogonal polynomials on the interval $[-h,h]$ with $p_k$ of degree $k$.  Such a set of polynomials is a \textbf{basis} for the vector space of degree at most $n$ polynomials on $[-h,h]$.
    \item If $\{p_0(x),p_1(x),...,p_k(x),...,p_n(x)\}$ is an orthogonal set of polynomials on $[-h,h]$ and if degree $p_k = k$ then $p_k$ has $k$ distinct roots in the interval $(-h,h)$.
    \item The set of \textbf{Legendre polynomials} $\displaystyle p_k(x) = \frac{1}{2^kk!}\frac{d^k}{dx^k}\left[(x^2-1)^k\right]$ forms such an orthogonal basis (on $[-1,1]$).
\end{itemize}
\end{tcolorbox}
\begin{enumerate}[resume=classQ]
    \item Set the $n$ values of $x_i$ to be the roots of $p_n(x)$.  Using polynomial long division, $x^j = S(x)p_n(x) + R(x)$ where $j \leq 2n-1$, $S(x)$ is degree at most $n-1$ and $R(x)$ is the remainder so also degree at most $n-1$.
    \begin{parts}
    \item Write $S(x) = \sum\limits_{i=1}^{n-1} a_i p_i(x)$.  How do you know it is possible to write $S(x)$ in this way?
    \vspace{1cm}
    \item Find $\int_{-1}^1 S(x)p_n(x)dx$.  \emph{By orthogonality, you know $\int_{-1}^1 p_i(x)p_n(x)dx$.}
    \vspace{1in}
 
 You should find that $\int_{-1}^1 x^j dx = \int_{-1}^1 R(x)dx$.  $R(x)$ is of degree $n-1$ or less, so an interpolating polynomial on $n$ points will exactly match it.  
 
 \item Argue that $\int_{-1}^1 x^j dx$ is equal to the quadrature rule applied to $R(x)$, so $\sum\limits_{i=1}^n c_i R(x_i)$.
 \vspace{1cm}
 
 \item Now consider the quadrature rule applied to $x^j = S(x)p_n(x) + R(x)$.  For $x_i$ the roots of $p_n(x)$ simplify $S(x_i)p_n(x_i) + R(x_i)$.
 \vspace{1cm}
    \end{parts}
\end{enumerate}
\begin{tcolorbox}
For any polynomial $P(x)$ of degree $\leq 2n-1$, $P(x) = S(x)p_n(x) + R(x)$.  $\displaystyle\int_{-1}^1 P(x)dx = \int_{-1}^1 R(x)dx = \sum\limits_{i=1}^n c_i R(x_i)$ where the $x_i$ are the $n$ roots of $p_n(x)$.

The Gaussian quadrature has a degree of precision of $2n-1$.
\end{tcolorbox}


\subsection*{Adaptive quadrature}

\begin{lstlisting}
% From Sauer: program 5.2 Adaptive Quadrature
% Computes approximation to definite integral
% Inputs: Matlab function f, interval [a0,b0], 
%   error tolerance tol0
% Output: approximate definite integral
function sum = adapquad(f,a0,b0,tol0)
sum=0; 
n=1; 
a(1)=a0; 
b(1)=b0; 
tol(1)=tol0; 
app(1)=trap(f,a,b);
% n is current position at end of the list
while n>0           
  c = (a(n)+b(n))/2;
  oldapp = app(n);
  app(n) = trap(f,a(n),c);
  app(n+1) = trap(f,c,b(n));
  if abs(oldapp-(app(n)+app(n+1))) < 3*tol(n)
    % success
	sum = sum + app(n) + app(n+1);
	% done with interval
	n = n-1;                         
  else
    % divide into two intervals
    % set up new intervals
	b(n+1)=b(n); 
	b(n)=c;
	a(n+1)=c;
	tol(n)=tol(n)/2; tol(n+1)=tol(n);
	% go to end of list, repeat
	n=n+1;                         
  end
end

function s=trap(f,a,b)
s = (f(a)+f(b))*(b-a)/2;

\end{lstlisting}

\begin{enumerate}[resume=classQ]
\item Let $I = \displaystyle\int_0^1 x^2 dx$.  For your reference, $I = 1/3$.
\begin{parts}
\item Call \texttt{adapquad(f,0,1,0.01)}.  In the first column, follow each of the variables on the first pass through the loop.  As a variable updates, cross out the old value and write the new one.

Trace the program for the first time through the loop.

Then trace again for the second pass through the loop.

\begin{tabular}{l|p{5cm} | p{5cm}| c}
variable & time 1 & time 2 & ...\\
\hline
& && \\
sum & && \\
& &&\\
n & && \\ & &&\\
a & && \\ & &&\\
b & && \\ & &&\\
tol & & &\\ & &&\\
app & && \\ & &&\\
c & && \\ & &&\\
oldapp & && \\%& &&\\

\end{tabular}

\item What does this code do?  Why is the method called \emph{adaptive}?
\vspace{1in}

\end{parts}
\end{enumerate}

\end{document}