\documentclass[12pt,letterpaper,noanswers]{exam}
\usepackage[usenames,dvipsnames,svgnames,table]{xcolor}
\usepackage[margin=0.9in]{geometry}
\renewcommand{\familydefault}{\sfdefault}
\usepackage{multicol}
\pagestyle{head}
\header{AM 111 Class 02}{}{Floating point, p.\thepage}
\runningheadrule
\headrule
\usepackage{siunitx}
\usepackage{enumitem}
\usepackage{graphicx} % more modern
\usepackage{amsmath} 
\usepackage{amssymb} 
\usepackage{hyperref}
%\usepackage{tcolorbox}
% % Python environment
% \lstnewenvironment{python}[1][]
% {
% \pythonstyle
% \lstset{#1}
% }
% {}

% % Python for external files
% \newcommand\pythonexternal[2][]{{
% \pythonstyle
% \lstinputlisting[#1]{#2}}}

% % Python for inline
% \newcommand\pythoninline[1]{{\pythonstyle\lstinline!#1!}}
% \newcommand{\vc}[1]{\underline{#1}}


\usepackage[most]{tcolorbox}
\usepackage{listings}

\definecolor{white}{rgb}{1,1,1}
\definecolor{mygreen}{rgb}{0,0.4,0}
\definecolor{light_gray}{rgb}{0.97,0.97,0.97}
\definecolor{mykey}{rgb}{0.117,0.403,0.713}

\tcbuselibrary{listings}
\newlength\inwd
\setlength\inwd{1.3cm}
% https://tex.stackexchange.com/questions/340700/ipython-notebook-input-and-output-cells-with-listings
\newcounter{ipythcntr}
\renewcommand{\theipythcntr}{\texttt{[\arabic{ipythcntr}]}}

\newtcblisting{pyin}[1][]{%
  sharp corners,
  enlarge left by=\inwd,
  width=\linewidth-\inwd,
  enhanced,
  boxrule=0pt,
  colback=light_gray,
  listing only,
  top=0pt,
  bottom=0pt,
  overlay={
    \node[
      anchor=north east,
      text width=\inwd,
      font=\footnotesize\ttfamily\color{mykey},
      inner ysep=2mm,
      inner xsep=0pt,
      outer sep=0pt
      ] 
      at (frame.north west)
      {\refstepcounter{ipythcntr}\label{#1}In \theipythcntr:};
  }
  listing engine=listing,
  listing options={
    aboveskip=1pt,
    belowskip=1pt,
    basicstyle=\footnotesize\ttfamily,
    language=Python,
    keywordstyle=\color{mykey},
    showstringspaces=false,
    stringstyle=\color{mygreen}
  },
}
\newtcblisting{pyprint}{
  sharp corners,
  enlarge left by=\inwd,
  width=\linewidth-\inwd,
  enhanced,
  boxrule=0pt,
  colback=white,
  listing only,
  top=0pt,
  bottom=0pt,
  overlay={
    \node[
      anchor=north east,
      text width=\inwd,
      font=\footnotesize\ttfamily\color{mykey},
      inner ysep=2mm,
      inner xsep=0pt,
      outer sep=0pt
      ] 
      at (frame.north west)
      {};
  }
  listing engine=listing,
  listing options={
      aboveskip=1pt,
      belowskip=1pt,
      basicstyle=\footnotesize\ttfamily,
      language=Python,
      keywordstyle=\color{mykey},
      showstringspaces=false,
      stringstyle=\color{mygreen}
    },
}
\newtcblisting{pyout}[1][\theipythcntr]{
  sharp corners,
  enlarge left by=\inwd,
  width=\linewidth-\inwd,
  enhanced,
  boxrule=0pt,
  colback=white,
  listing only,
  top=0pt,
  bottom=0pt,
  overlay={
    \node[
      anchor=north east,
      text width=\inwd,
      font=\footnotesize\ttfamily\color{mykey},
      inner ysep=2mm,
      inner xsep=0pt,
      outer sep=0pt
      ] 
      at (frame.north west)
      {\setcounter{ipythcntr}{\value{ipythcntr}}Out#1:};
  }
  listing engine=listing,
  listing options={
      aboveskip=1pt,
      belowskip=1pt,
      basicstyle=\footnotesize\ttfamily,
      language=Python,
      keywordstyle=\color{mykey},
      showstringspaces=false,
      stringstyle=\color{mygreen}
    },
}





%\newcommand{\note}[1]{\textcolor{red}{[#1]}} % show notes in red
 \newcommand{\note}[1]{} % don't display notes

\begin{document}
 \pdfpageheight 11in 
  \pdfpagewidth 8.5in

\noindent 

\note{calendar:
\begin{enumerate}
    \item Tu intro PS01
    \item Th binary PS01
    \item Tu condition number PS01/2
    \item Th least sq PS02
    \item Tu lin alg PS02/3
    \item Th lin alg PS03
    \item Tu least sq PS03/4
    \item Th ?? PS04
    \item Tu root finding PS04/5
    \item Th root finding PS05 (early)
    \item Tu integration PS06
    \item Th quiz
    \item Tu interpolation PS06
    \item Th interpolation PS06
    \item Tu integration PS07
    \item Th Monte Carlo PS07
    \item Tu differentiation PS08
    \item Th differentiation PS08
    \item Tu diff eq
    \item Th application of diff eq
    \item Tu ODEs
    \item Th ODEs
    \item Tu neural nets
    \item Tu neural nets
    \item Th quiz
    \item Tu presentations
\end{enumerate}}

\note{
\begin{itemize}
    \item what is a floating point system
    \item example
    \item IDing info about a floating point system
    \item 
\end{itemize}
}
\setcounter{section}{-1}
\section{Preliminaries}
\begin{itemize}
\itemsep0pt
\item There will be a skill check in class during the next class.  The problem info is below.
\item Problem set 01 will be assigned Friday and due the following Friday.
\item OH this week: Thursday from 1-2pm in Pierce 316 (Sarah's office)
\end{itemize}

\hrule
\vspace{0.2cm}


\noindent\textbf{Big picture}
\begin{itemize}
    \itemsep0pt
\item How are numbers represented in computers?  What implications does this have for solving problems?

\item Why do $3/2 - 1/2 - 1$ and $7/3 - 4/3 - 1$ give different results?
\begin{itemize}
    \itemsep0pt
    \item How is a real number represented in the computer?
    \item Where does error appear in that representation?
    \item How does that impact subtraction?
\end{itemize}
\end{itemize}

\vspace{0.2cm}
\hrule
\vspace{0.2cm}

\noindent \textbf{Skill check practice}
Given a binary number, round it to nearest, keeping 4 bits after the radix point.  The skill check will avoid the special case where the truncated part is exactly $0.00001$.

\noindent Example:

Round $(+1.00111001)_2$ to nearest, keeping 4 bits after the radix point. 


\vspace{0.2cm}
\hrule
\vspace{0.2cm}

\noindent \textbf{Skill check solution}

We check the 5th bit after the radix point to decide whether to round down or up.  Bit 5 is a $1$ so we round up.  

To round up, add $0.0001$ to the chopped value.

$1 + 1 = 10$ in binary, and $11 + 1 = 100$, so 
\begin{align*}
    &+1.0011 \\
    &+0.0001 \\
    \cline{1-2}
     &+1.0100
\end{align*}

Note: if we needed to round down, we would truncate, keeping the first 4 bits and dropping the additional bits.

\vspace{0.2cm}
\hrule
\vspace{0.2cm}

Additional notes
\begin{itemize}
\item Problem sets have a python notebook and a pdf part
\item shows up as two assignments on gradescope
\item github has a week01.ipynb with working examples of for loop, while loop, various other things
\item You're asked to make comments in a markdown cell in a couple of places.
\end{itemize}

\vspace{0.2cm}
\hrule
\vspace{0.2cm}

\section{Representing numbers in a computer}
\subsection{Converting to binary (Sauer \S0.2)}
\textbf{Algorithm for base 10 integer to binary:}
\begin{tcolorbox}


In base $10$ when you divide an integer by $10$, the remainder is the right most digit of the integer: $17/10 = 1\ R7$ where $7$ is the right most digit.

We can continue to divide by $10$ and read off the remainder as the next digit.  For a decimal number, we would do that until we have a single digit and a remainder

By listing the digit and then the remainders in the right order, you can reconstruct the number. \\


Something similar works in binary.

When you divide an integer by $2$ the remainder is a $0$ or a $1$:
$17/2 = 8 R1$ where $1$ will be the right most bit of the binary representation.


\end{tcolorbox}

Example:

$17/2 = 8 R1$.  $(17)_{10} = (\_ \_ \_ \_ 1)_2$.

$8/2 = 4 R0$.  $(17)_{10} = (\_ \_ \_ 0 1)_2$.

$4/2 = 2 R0$.  $(17)_{10} = (\_ \_ 0 0 1)_2$.

$2/2 = 1 R0$.  $(17)_{10} = (1 0 0 0 1)_2$.



\textbf{Algorithm for base 10 fractional or decimal part to binary:}

\begin{tcolorbox}
    To convert an integer to binary, we divided by $2$ and used the remainder.  To convert a decimal or fractional part to binary, we will multiply by $2$ and read off the first digit.
\end{tcolorbox}

Example:

Find $\dfrac{1}{3}$ in binary.

$\dfrac{1}{3} = (0. \_ \_ \_ \_)_2$.

$\dfrac{1}{3}*2 = \dfrac{2}{3}$.    $\dfrac{1}{3} = (0.0 \_ \_ \_)_2$.

$2/3 * 2 = 4/3 = 1 + 1/3$.  $\dfrac{1}{3} = (0.01\_ \_)_2$. 

The fractional part is now $\dfrac{1}{3}$. 

Wait... we've already done $1/3$ so we know what happens from here.  This is a repeating binary expansion.  

$\dfrac{1}{3} = (0.\overline{01})_2$

\begin{enumerate}[resume]
    \item Use division by $2$ and keep the remainders to find $27$ in binary.
    \item Use multiplication by $2$ and keep the integer parts to find $0.3$ in binary.
    \item Convert $27.3$ to binary.
\end{enumerate}



% \eject

% \noindent\textbf{Some answers}
% \begin{enumerate}
%     \item a) $+9999\times 10^5$.  b) $+9.999\times 10^5$.  c) $243$.  d) $-3.400\times 10^{-3}$.  e) $+1.000\times 10^0$.  f) $1.001\times 10^0$ 
%     \item a) $1.001\times 10^0$ is the next number, so $\epsilon = 1.000\times10^{-3}$.  b) $\pi = 3.141592...$ so $3.142$ (round) or $3.141$ (truncate) c) $+1.000\times10^{-5}$ so $0.00001$. d) $0 = 0.0000$ but there has to be a nonzero number to the left, so not clear how to, no.
%     \item a) $+2.000\times10^{-5}$ b) $1.000 + 0.00001 = 1.00001$ but we stop at $4$ digits so $+1.000\times 10^0$.  c) $-1.000 + 0.00001$ is $-1.000$ so $0$.
%     \item \begin{tabular}{c|c|c}
%      & chop & nearest \\
%      \hline
%   2.394   & 2.3& 2.4\\
% 2.251 & 2.2& 2.3 \\
% 2.250 & 2.2 & 2.2 \\
% \end{tabular}
% $2.25$ was between $2.2$ and $2.3$ so look to the $3$ to decide which way to round.
% \item chop: $\vert(2.394-2.3)/2.394\vert = 0.94/2.394$.  round: $\vert(2.394-2.4)/2.394\vert = 0.006/2.394$.  chopping is worse (something like 40\% !?)
% \item a) $(11)_2 = 1+2 = 3$. b) 4+0+1 = 5.  c) $5+ 1*2^{-1} 5.5$.
% \item a) $16 = 2^4$ so $(10000)_2$. b) $17 = 16+1$ so $(10001)_2$.
% \item $27/2 = 13 R1$.  $13/2 = 6 R1$.  $6/2 = 3 R0$.  $3/2 = 1 R1$.  $(11011)_2$.  Check: $16+8+0+2+1 = 27.$
% \item $0.3*2 = 0.6$ so $(0.0...)_2$.  $0.6*2 = 1.2$ so $(0.01...)_2$.  $0.2*2 = 0.4$ so $(0.010...)_2$.  $0.4*2 = 0.8$ so $(0.0100...)_2$.  $0.8*2 =1.6$ so $(0.01001...)_2$.  We have seen $0.6$ onward before so $1001$ will repeat.  $(0.0\overline{1001})_2$.  Check: $x = (0.0\overline{1001})_2$.  $2x = (0.\overline{1001})_2$.  $2*2^4x = (1001.\overline{1001})_2$.  $(1001)_2 = 9$ so $2x + 9 = 32x$.  $30x = 9 \Rightarrow x = 3/10 = 0.3$
% \item a) $27.3 = 27+0.3 = (11011.0\overline{1001})_2$  b) $+1.1011010011001100\times 2^4$ <-- 12 bits after the radix point. $12+8*5 = 52$ so make 5 copies of the last 8:

% $+1.10110100110011001100110011001100110011001100110011001100\times 2^4$.  For rounding, the next bits are $11$ so round up

% $+1.\framebox[11.7cm][l]{10110100110011001100110011001100110011001100110011001101}\times 2^4$
% \end{enumerate}



\subsection{Floating point (Sauer \S 0.3)}

So far we have not done any approximation, and we have not made our binary representations finite, so that they can fit in computer memory.

To store a real number in the computer we use a \textbf{floating point system}.

\begin{tcolorbox}
\begin{itemize}
\itemsep0pt
    \item A \textbf{floating point number} consists of three parts: the \textbf{sign} of the number, a string of bits called the (\textbf{mantissa}), and an \textbf{exponent}.  \cite{sauer2018numerical}
    \item These parts are stored together.
    \item A floating point system can be characterized by four numbers:
\begin{itemize}
\itemsep0pt
    \item $\beta$ (base)
    \item $N+1$ (precision)
    \item $[L,U]$ (range for exponents)
\end{itemize}

\item A floating point number is of the form \[ \pm \left(d_0 + d_1\beta^{-1} + d_2 \beta^{-2} + ... + d_{N}\beta^{-N}\right) \beta^p\] where $0\leq d_i\leq \beta-1$ and $L\leq p \leq U$ are integers.

In base $10$ you would write $d_0.d_1d_2...d_{N} \times 10^p$.
\item $\pm$ is the sign, $d_0d_1d_2...d_{N}$ is the mantissa, $p$ is the exponent, and $N+1$ is the precision.
\item In a \textbf{normalized} system we require $d_0\neq 0$.  In binary, $d_0 = 1$.
\end{itemize}
\end{tcolorbox}

\textbf{Practice}

$(11011.1)_2$ becomes $+1.10111\times 2^4$.

$(3749.53)_{10}$ becomes $+3.74953\times 10^3$





\vspace{0.2cm}
\hrule
\vspace{0.2cm}

\subsection{Shortening the numbers: rounding vs chopping}
\begin{tcolorbox}
\begin{itemize}
\itemsep0pt
    \item Let $\text{fl}(x)$ denote the floating point approximation to $x$.
    \item To convert from $x$ to $\text{fl}(x)$ one option is to \textbf{chop}, taking the first $p$ digits.  
    
    Dropping the $N+1$st digit leads to a relative error of $\left\vert\dfrac{\text{fl}(x) - x}{x}\right\vert\leq \beta^{-(p-1)}$
    \item Another option is \textbf{rounding to nearest} by taking the floating point number nearest to $x$.  
    
    This leads to a relative error of $\left\vert\dfrac{\text{fl}(x) - x}{x}\right\vert\leq \frac{1}{2}\beta^{-(p-1)}$
    
    Note: If $x$ is equidistant between two floating point numbers (think of $1.5$ when rounding to the nearest integer), then the value of $d_{p-1}$ is used to determine whether to round up or down.  
    \item The maximum relative error gives the accuracy of the floating point system.  It is \[\epsilon_{\text{mach}} = \max\limits_{x\in[\text{UFL,OFL}]} \left\vert\dfrac{\text{fl}(x) - x}{x}\right\vert.\]
\end{itemize}
\end{tcolorbox}

\noindent\textbf{Example}.




\vspace{0.2cm}
\hrule
\vspace{0.2cm}
\noindent\textbf{Floating point in computers}

\begin{tcolorbox}
\begin{itemize}
\itemsep0pt
    \item The floating point systems used in computers typically are in base $@$(binary), so the digits of the floating point numbers, $d_i = 0$ or $1$.  We will use $b_i$ interchangeably with $d_i$ when we are working in binary.
    \item A single binary digit is called a \textbf{bit}.
    \item In double precision, a floating point number has $64$ bits.  \[s e_1e_2...e_{11}b_1b_2...b_{52}\] where $s$ is the sign, followed by $11$ bits for the exponent, and then the $52$ bits following the decimal point (also called the \textbf{binary point} or \textbf{radix point}).
    \item A few exceptional numbers also need to be represented: NaN (not a number, i.e. 0/0), Inf (infinity, i.e. $1/0$), 0 (note that $d_0 = 1$ so $0$ needs a special representation)
\end{itemize}
\end{tcolorbox}

Where does the error enter?

Why could it cause problems in subtractions?

Next time: 3/2 - 1/2 - 1

7/3 - 4/3 - 1 is on the problem set.

need to review 1s complement...

\vspace{0.2cm}
\hrule
\vspace{0.2cm}




% \bibliographystyle{plain}
% \bibliography{AM111references.bib}


\end{document}