\documentclass[12pt,letterpaper,noanswers]{exam}
%\usepackage{color}
\usepackage[usenames,dvipsnames,svgnames,table]{xcolor}
\usepackage[margin=0.9in]{geometry}
\renewcommand{\familydefault}{\sfdefault}
\usepackage{multicol}
%\newcommand{\mathbf}[1]{\boldsymbol{#1}}
\pagestyle{head}
\definecolor{c02}{HTML}{FFBBBB}
\definecolor{c03}{HTML}{FFDDDD}
\header{AM 111 Problem Set 06}{}{{\colorbox{c02}{\makebox[2.8cm][l]{Due Fri Oct 27}}}\\at 12pm}
\runningheadrule
\headrule
\usepackage{diagbox}
\usepackage{graphicx} % more modern

\usepackage{amsmath} 
\usepackage{amssymb} 

\usepackage{hyperref}

\usepackage{tcolorbox}
\usepackage[framed,numbered,autolinebreaks,useliterate]{mcode}


\def\been{\begin{enumerate}}
\def\enen{\end{enumerate}}
\def\beit{\begin{itemize}}
\def\enit{\end{itemize}}
\def\dsst{\displaystyle}
\def\dx{\Delta x}
\hyphenation{}
\newcommand{\blank}[1]{\underline{\hspace{#1}}}

\makeatletter
\newcommand{\pyf}{%
  \begingroup\catcode`_=12
  \pyf@
}
\newcommand{\pyf@}[1]{\texttt{#1}\endgroup}
\makeatother

\newcommand{\note}[1]{\textcolor{red}{#1}} % show notes in red
%\renewcommand{\note}[1]{} % don't display notes


\begin{document}
 \pdfpageheight 11in 
  \pdfpagewidth 8.5in

\begin{itemize}
    \itemsep0pt
    \item Find the \href{https://github.com/sarah1123/ScientificComputing-APMTH111/blob/main/2023Fall/PythonFiles/06_quadrature/ProblemSet06.ipynb}{ProblemSet06 Python template} via this link.
    \item Use this python notebook for all programming work on the problem set.  Submit the notebook to the Python assignment on Gradescope.
    \item Submit the other problems to the pdf assignment on Gradescope.
    \item Late work: Problem sets are due at noon on Fridays.  They are accepted until 5pm for all students without penalty.  In addition, you have three 29-hour late days that allow you to submit until 5pm on Saturday.  You don't need to ask to use your late days, just keep track of them for yourself.
\end{itemize}


\begin{questions}
\item Submit this via the "Short Project Proposal" assignment on Gradescope.

The goal of the final project is for you to apply numerical analysis or scientific computing techniques to an applied or theoretical problem, or to conduct an investigation into a method not discussed in class.

{\large You are encouraged to talk with the course staff as you work to develop your initial project ideas.}

\begin{enumerate}
\itemsep0pt
      \item A (short) preliminary project proposal is due as part of problem set 06.
    \item You will propose two projects.
    \item Each students will work on a different project, with each project based on a different paper or resource.
\end{enumerate}

\noindent For each preliminary project proposal, answer the following questions:
\begin{itemize}
\itemsep0pt
\item What will be the aim or purpose of the project?
\item What paper(s) or resource(s) would you like to base your project work around?
\item How does the intended project incorporate mathematical topics related to this course?
\item What numerical method or methods will you implement or study as part of the project?
\item What do you think you will you need to learn to be able to complete your project?
\end{itemize}

\item (Heath 8.10) To create a Newton-Cotes quadrature rule, we fixed the nodes and determined weights.

In a \emph{Chebyshev} quadrature rule, the weights are all set to the same value, $w$, which eliminates $n$ multiplications, and the nodes are determined. 

For a three point Chebyshev rule on $[-1,1]$ there are four unknowns: the three nodes, $x_1, x_2, x_3$ and the weight $w$.
\begin{parts}
\item 
Find node positions and weights that maximize the degree of precision of the rule.  
\item What degree of precision did you find?
\item If you think of the integral as $2\langle f \rangle$ where $2$ is the length of the interval and $\langle f \rangle$ is an estimate of the average value of the function on the interval, what value would you expect for the weights?  Does this match what you found?
\end{parts}

\item (trapezoid and Simpson's)
\begin{parts}
\item Implement \pyf{trap(a,b,f)}, a function that returns the trapezoid rule estimate of the integral of $f$ on the interval $[a,b]$.
\item Implement \pyf{simpsons(a,b,f)}, a function that returns the Simpson's rule estimate of the integral of $f$ on the interval $[a,b]$.
\item Check the text for error information for trapezoid rule and Simpson's rule.

If you wish to obtain a more accurate approximation of an integral, which will gain more accuracy: dividing the interval in half and using trapezoid rule on each subinterval, or using Simpson's rule on the entire interval?  Support your answer by considering the error.

\item Using the functions you wrote, compare the split trapezoid rule to Simpson's rule for $\int_0^{\pi} x^2\sin x dx$, $\int_0^{2\sqrt{3}} \frac{1}{\sqrt{x^2+4}}dx$.  Compare with the exact values of the integrals (feel free to use Mathematica for that).
\end{parts}

\item (composite rules)
\begin{parts}
\item Implement \pyf{composite_trap(a,b,f,m)}, a function that returns a composite trapezoid rule estimate of the integral of $f$ on the interval $[a,b]$ using $m$ panels.
\item Implement \pyf{composite_simpsons(a,b,f,m)}, a function that returns a composite Simpson's rule estimate of the integral of $f$ on the interval $[a,b]$ using $m$ panels.
\item (Sauer \S 5.2 9) For the two integrals you computed in the problem above, calculate the approximation error of the composite trapezoid rule for $m = 1, 2, 4, ..., 2^8$.  Make a log-log plot of $\log \text{error}$ vs $\log h$, where $h$ is the panel width.
\item (Sauer \S 5.2 10) For the two integrals you computed in the problem above, calculate the approximation error of the composite Simpson's rule for $m = 1, 2, 4, ..., 2^8$.  Make a log-log plot of $\log \text{error}$ vs $\log h$, where $h$ is the panel width.
\item What is the slope of each error plot?  How does that compare to the theory in each case?

\emph{Note that according to theory, $E = k h^n$ so $\log E = n\log h + \log k$.}
\end{parts}


% \item (Richardson extrapolation and Romberg integration)

% Romberg integration is a method where Richardson extrapolation is applied multiple times to develop a higher order (lower error) integration method.

% Use the composite trapezoid rule, \[\displaystyle\int_a^b f(x) dx \approx \frac{h}{2}\left(y_0 + 2\sum\limits_{k=1}^{m-1}y_k  + y_m\right).\]

% An exact expression for the integral is given by 
% \[\displaystyle\int_a^b f(x) dx = \frac{h}{2}\left(y_0 + 2\sum\limits_{k=1}^{m-1}y_k  + y_m\right) + c_2h^2+c_4h^4+c_6h^6+...\] where $c_k$ depend on higher derivatives of $f$ at $a$ and $b$, but not on $h$.

% \begin{parts}
% \item Let $h_1 = b-a$ and $h_2 = h_1/2$.

% Let $R_{11}$ be the trapezoid rule with $h = h_1$ and $R_{21}$ be the composite trapezoid rule with $h = h_2$.

% We have 
% \[\int_a^b f(x) dx = R_{11} + c_2h_1^2 + c_4h_1^4 + c_6h_1^6+...\]
% \[\int_a^b f(x) dx = R_{21} + c_2h_2^2 + c_4h_2^4 + c_6h_2^6+...\]

% Use Richardson extrapolation to combine $R_{11}$ and $R_{21}$ to form a new method.  Call this new method $R_{22}$.
% \item Show that $R_{22}$ is Simpson's rule (with step size $h_2$).
% \item Using $h_2$ and $h_3 = h_2/2$ you can similarly construct $R_{31}$ (the composite trapezoid rule with $h = h_3$) and $R_{32}$ (a combination of $R_{21}$ and $R_{31}$).

% Use extrapolation on $R_{22}$ and $R_{32}$ to form a new method.  Call this new method $R_{33}$.

% \item $R_{21} = \frac{1}{2}R_{11}+h_2 f(a+h_2)$.  Find an analogous formula relating $R_{31}$ and $R_{21}$.
% \end{parts}

% \item (based on Koumoutsakos et al notebook 7.1)

% Jet engines produce contrails (the thin white clouds that are left behind as a plane passes through the sky).

% It is possible to model the velocity profile across the contrail. Assume the velocity of exhaust leaving the engine is $v_{\text{exhaust}} = 1200$ km/h (at the center of the engine).  Assume the velocity of the air surrounding the jet is $v_{\text{freestream}} = 800$ km/h.

% Let the jet engine have radius $R$ and set $r$ to be the distance to the center.  

% As it leaves the jet, the velocity profile in a circular cross section of the contrail can be modeled by $v(r) = v_{\text{exhaust}} + (v_{\text{freestream}}-v_{\text{exhaust}})\text{erf}(2r/R)$

% $\text{erf}(x)$ is the error function, given by $\displaystyle\text{erf}(x) = \frac{2}{\sqrt{\pi}}\int_0^x e^{-t^2}dt$

% Following the steps below, use Romberg integration to find the velocity profile.
% \begin{parts}


% \item Create two functions: \texttt{def f(x):}, the function you will integrate, and \texttt{def romberg(a,b,f):}, a function that will perform Romberg integration of a function $f$ on the interval $[a,b]$.

% The output of \texttt{romberg} should be a list or matrix of approximations to the integral, $R_{11}, R_{21}, R_{31}, R_{22}, R_{32}, R_{33}$.

% \emph{To the extent possible reuse your function evaluations from generating $R_{11}$ to create $R_{21}$ and from $R_{21}$ to create $R_{31}$.}

% \item To test your code, compute the approximation $R_{33}$ for the integral $f(t) = \frac{2}{\sqrt{\pi}}e^{-t^2}$ on the interval $[0,0.5]$.  Output all of the $R_{jk}$ calculated on the way, along with the associated error.  Use \texttt{erf\_true= 0.52049987781304653} as your exact value.  

% \item Define a function \texttt{def v(x):} that takes in a value of $r/R$, uses \texttt{romberg} for integration, and returns the velocity given by the model.

% Create and plot the velocity profile along the radial direction, $v(r)$, with $r/R \in[0,1]$ used as the horizontal axis.  

% Comment on whether the profile makes sense physically, especially at $\frac{r}{R} = 0$ and $\frac{r}{R} = 1$.
% \end{parts}

% % \item applied problem: brakes?  corrugation?

\item \emph{time permitting} (adaptive quadrature) %(Batman curve area)

%You'll use Simpson's rule to find the area of the Batman logo.  

Let $S_{[a,b]}$ denote Simpson's rule on the interval $[a,b]$.  \[\displaystyle\int_a^b f(x)dx = S_{[a,b]}-\frac{1}{90}h^5f^{(iv)}(c_0),\] where $E_{[a,b]}=-\frac{1}{90}h^5f^{(iv)}(c_0)$ is the error when Simpson's rule on $[a,b]$ is used to approximate the integral.

\begin{parts}
% \item (optional) Provide a short back story for how finding the area of the Batman logo will allow you to save Gotham City.
\item Let $c = (a+b)/2$.  Find $E_{[a,c]+[c,b]}$, the error when when $S_{[a,c]} + S_{[c,b]}$ is used to approximate the integral.

You'll initially find two contributions to the error and can combine them into a single term via the intermediate value theorem (stated just below).

\emph{The intermediate value theorem}: for $g$ a continuous function on $[a,b]$, $x_1, x_2$ points in $[a,b]$, and $a_1, a_2>0$, there exists a number $c$ between $a$ and $b$ such that $a_1g(x_1) + a_2g(x_2) = (a_1+a_2)g(c)$.


\item $S_{[a,b]}$ and $S_{[a,c]}+S_{[c,b]}$ are both approximations for the same integral.  \[S_{[a,b]}-\left(S_{[a,c]}+S_{[c,b]}\right) \approx \frac{15}{16}\frac{1}{90}h^5f^{(iv)}(c) \approx k\vert E_{[a,c]+[c,b]}\vert\] for some constant $k$.  Find $k$.

%\item Create a function \texttt{simpson(a,b,f):} that takes in an interval $[a,b]$ and a function $f$ and applies Simpson's rule on the interval.  \emph{This is not the composite rule, but Simpson's rule applied to a single panel.}

\item Using your \pyf{simpsons} function, approximate $\displaystyle\int_1^3 \frac{1}{x} dx$ using $S_{[a,c]}+S_{[c,b]}$ and provide an estimate of the absolute value of the error.  Use $k = 10$ for your estimate.  


\emph{$k=10$ is more conservative than the $k$ you found in $b$.}

\item Compare your estimate to the actual error.

\end{parts}

\question Reflection
\begin{parts}
\item When you worked on the problem set where did you get stuck or become confused?
\item What aspects of the course challenged you this week?  What did you do to address those challenges?  What topics/ideas/procedures do you not yet understand?
\item What did you understand the best this week?  What, if anything, do you understand better this week than you did in the past?
\item List the people that you worked with or consulted on the problem set problems.  This might include other students in the course, course instructors, or people who have previously taken the course.
\item Below, indicate how much of your time for this class has been doing the following activities:
	\begin{enumerate}
	\item Working on the problem set (including time in Python)
	\item Reviewing course materials, including the course textbooks
	\item Working through supplementary materials
	\item Going to office hours or lab
	\item Other (please specify)
	\end{enumerate}
\item If you used chatGPT or other AI tools, attach that information as part of this question.
\end{parts}

\end{questions}

\end{document}