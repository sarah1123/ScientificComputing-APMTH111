\documentclass[12pt,letterpaper,noanswers]{exam}
%\usepackage{color}
\usepackage[usenames,dvipsnames,svgnames,table]{xcolor}
\usepackage[margin=0.9in]{geometry}
\renewcommand{\familydefault}{\sfdefault}
\usepackage{multicol}
%\newcommand{\mathbf}[1]{\boldsymbol{#1}}
\pagestyle{head}
\definecolor{c02}{HTML}{FFBBBB}
\definecolor{c03}{HTML}{FFDDDD}
\header{AM 111 Problem Set 07}{}{{\colorbox{c02}{\makebox[2.8cm][l]{Due Fri Nov 3}}}\\at 12pm}
\runningheadrule
\headrule
\usepackage{diagbox}
\usepackage{graphicx} % more modern

\usepackage{amsmath} 
\usepackage{amssymb} 

\usepackage{hyperref}

\usepackage{tcolorbox}
\usepackage[framed,numbered,autolinebreaks,useliterate]{mcode}


\def\been{\begin{enumerate}}
\def\enen{\end{enumerate}}
\def\beit{\begin{itemize}}
\def\enit{\end{itemize}}
\def\dsst{\displaystyle}
\def\dx{\Delta x}
\hyphenation{}
\newcommand{\blank}[1]{\underline{\hspace{#1}}}

\makeatletter
\newcommand{\pyf}{%
  \begingroup\catcode`_=12
  \pyf@
}
\newcommand{\pyf@}[1]{\texttt{#1}\endgroup}
\makeatother

\newcommand{\note}[1]{\textcolor{red}{#1}} % show notes in red
%\renewcommand{\note}[1]{} % don't display notes


\begin{document}
 \pdfpageheight 11in 
  \pdfpagewidth 8.5in

\begin{itemize}
    \itemsep0pt
    \item Find the \href{https://github.com/sarah1123/ScientificComputing-APMTH111/blob/main/2023Fall/PythonFiles/07_quadrature/}{ProblemSet07 Python template} via this link.
    \item Use this python notebook for all programming work on the problem set.  Submit the notebook to the Python assignment on Gradescope.
    \item Submit the other problems to the pdf assignment on Gradescope.
    \item Late work: Problem sets are due at noon on Fridays.  They are accepted until 5pm for all students without penalty.  In addition, you have three 29-hour late days that allow you to submit until 5pm on Saturday.  You don't need to ask to use your late days, just keep track of them for yourself.
\end{itemize}


\begin{questions}
% \item Submit this via the "Short Project Proposal" assignment on Gradescope.

% The goal of the final project is for you to apply numerical analysis or scientific computing techniques to an applied or theoretical problem, or to conduct an investigation into a method not discussed in class.

% {\large You are encouraged to talk with the course staff as you work to develop your initial project ideas.}

% \begin{enumerate}
% \itemsep0pt
%       \item A (short) preliminary project proposal is due as part of problem set 06.
%     \item You will propose two projects.
%     \item Each students will work on a different project, with each project based on a different paper or resource.
% \end{enumerate}

% \noindent For each preliminary project proposal, answer the following questions:
% \begin{itemize}
% \itemsep0pt
% \item What will be the aim or purpose of the project?
% \item What paper(s) or resource(s) would you like to base your project work around?
% \item How does the intended project incorporate mathematical topics related to this course?
% \item What numerical method or methods will you implement or study as part of the project?
% \item What do you think you will you need to learn to be able to complete your project?
% \end{itemize}

\item 

Project proposal

Your will have one (or more) short project proposals approved via Gradescope (I want to avoid two projects using the exact same resources).

This week, write a full project proposal.  Submit this via the "Full Project Proposal" assignment on Gradescope.  As part of your proposal, submit all of the information that is requested below.

\begin{parts}
    \item Aim: What will be the purpose of the project?  Create a defined question.  Go beyond generalities to include one or more specific goals.
    
    \emph{Your goal might be to understand and implement a method, to apply a method to a scientific problem, or something else that is of interest to you and related to our course.}
    \item Components: Identify the tasks that you will need to accomplish in order to satisfy your project goals.
    \item Knowledge building: Provide detailed information about how you will engage in knowledge building.  What tools and methods do you need to understand?  What resources will you use?
    \item Computational methods: What computational method(s) do you plan to implement yourselves?  What method(s) will you incorporate by using built-in options in Python?
%    \item Weekly progress: Provide expectations of weekly progress.
    \item Scope: If this turns out to be more (or less) work than you expected, how can you adjust the project to take on more (or less)?
    \item Create an intended timeline for your  work on the project.  Include information about what you plan to work on each week.

\end{parts}


% \item (adaptive quadrature: Batman logo)

% In adaptive quadrature

% On the previous problem set, you may have found (see question 5 on PSet 06) that \[S_{[a,b]}-\left(S_{[a,c]}+S_{[c,b]}\right) \approx k\vert E_{[a,c]+[c,b]}.\vert\]

% This enables you to estimate $\vert E_{[a,c]+[c,b]}.\vert$.

% \begin{parts}
% \item Modify the adaptive quadrature code from the class handout.  

% (1) Rewrite it in Python (where indices start at $0$ rather than $1$).  

% (2) Write it to use Simpson's rule rather than the trapezoid rule: $k=3$ for trapezoid rule.  Use your value from the previous problem set for Simpson's.  

% (3) Edit it to pass $f(a)$ and $f(b)$ to Simpson's, so as to avoid extra function evaluations (and keep track of those in variables \texttt{fa} and \texttt{fb}).

% \item Provide a reason why it is absolutely critical to find the area of the Batman logo in order to save Gotham from disaster.

% Using the \texttt{batman\_upper} and \texttt{batman\_lower} functions given (the logo extends from $x=-7$ to $x=7$), using a tolerance of $10^{-6}$, and a maximum depth of $100$, estimate the area of the Batman logo via adaptive quadrature.

% \item What was the maximum value of $n$ (a measure of the number of subintervals)?  What was the smallest subinterval?  If you were to use this smallest subinterval as your panel size for a composite method, how many panels would be needed?

% \item Ask WolframAlpha for ``area of batman equation'' (and use $a=1$ in the area formula that they return).  Compare this to your calculated value.  Did you actually meet the error tolerance that was specified?
% \end{parts}

\item (Monte Carlo)

For a $d$-dimensional ball, for $\mathbf{x} = (x_1, x_2, ..., x_d)^T$, $\Vert\mathbf{x}\Vert^2 < 1$ when $\mathbf{x}$ is inside of the ball.



\begin{parts}
\item Create a function \texttt{def chi(xd):} that takes a vector of points $\mathbf{x}$ and returns a vector of $0$s and $1$s: $0$ if a point in the vector is outside of the unit ball and $1$ if it is inside.


\item For the sphere in $3$-space of radius $1$ (a $3$-sphere, denoted $\mathbb{S}^2$), use the trapezoid rule to estimate the volume.

Use $11$ points along each dimension (and an interval of $0.2$).


$\displaystyle\int_{-1}^1\left(\int_{-1}^1\left(\int_{-1}^1 \mathcal{I}_{\Vert \mathbf{x}\Vert <1} dz \right)dy\right)dx$ where $\mathcal{I}_{\Vert \mathbf{x}\Vert <1} = \left\{\begin{array}{l}1 \text{ if } \Vert \mathbf{x}\Vert\leq 1 \\
0 \text{ otherwise}\end{array}\right.$

One way to do this is to think of the integral as an iterated integral: Let $g(x,y) = \displaystyle\int_{-1}^1 \mathbb{I}_{\Vert \mathbf{x}\Vert <1} dz$.  Compute $g(x,y)$ at the $11\times 11$ pairs of $(x,y)$ values that you are using.

Then let $f(x) = \displaystyle\int_{-1}^1 g(x,y) dy$ and compute $f(x)$ at the $11$ $x$ values that you are using.

Finally, find $\displaystyle\int_{-1}^1 f(x) dx$

Find the error by comparing to the volume, $\frac{4}{3}\pi$.

\item Use Monte-Carlo integration to estimate the volume to the same error.  Run your Monte-Carlo method multiple times.  Report the average number of samples you needed to meet the error bound.

\item For the $10$-sphere of radius $1$, the volume enclosed is $\pi^5/120$.

Use Monte-Carlo to estimate this, with the same percent error as your estimates had above.  Report the average number of samples needed.

\item If you had kept using $11$ gridpoints along each direction, how many gridpoints would you need for trapezoid rule?  How does that compare to the number of samples you needed for Monte-Carlo?


\end{parts}

\question (more quadrature)

from Greenbaum and Chartier.
\begin{parts}
    \item Consider the quadrature $\int_{-1}^1 f(x)\ dx\approx f(a) + f(-a)$.  For what values of $a$ (if any) will this formula have a degree of precision of at least $1$?
%    \item For the same approximation, for what values of $a$ (if any) will this formula have a degree of precision of at least $3$?
    \item Consider the quadrature $\int_0^1 f(x) dx \approx a_1 f(0) + a_2 f(1)$.  Choose $a_1$ and $a_2$ so that this formula is exact for all functions of the form $f(x) = b_1e^x + b_2\cos(\pi x/2)$ where $b_1$ and $b_2$ are constants.
\end{parts}

\item (adaptive quadrature: Batman logo)

On Problem Set 06, you may have found that \[S_{[a,b]}-\left(S_{[a,c]}+S_{[c,b]}\right) \approx k\vert E_{[a,c]+[c,b]}.\vert\]
where $c = (a+b)/2$.  Read problem 5 on problem set 06 to become familiar with the notation.  This expression is comparing Simpson's rule on a full interval to Simpson's rule used on two half-intervals.

This expression enables you to estimate the error from computing the integral using the two subintervals, $\vert E_{[a,c]+[c,b]}.\vert$.  The value of $k$ for Simpson's rule is $15$.

\begin{parts}
\item Find the adaptive quadrature code in the 'Week07.ipynb' file.  Edit it to use Simpson's rule instead of trapezoid.  \emph{Notice that it passes $f(a)$ and $f(b)$ to Simpson's to avoid extra function evaluations.}


\item Using the \texttt{batman\_upper} and \texttt{batman\_lower} functions given (the logo extends from $x=-7$ to $x=7$), using a tolerance of $10^{-6}$, and a maximum depth of $100$, estimate the area of the Batman logo via adaptive quadrature with Simpson's.

\item Provide a reason (of your choice) why it is absolutely critical to find the area of the Batman logo in order to save Gotham from disaster.


How many subinternals are used (choose the larger of the number for the upper and lower parts)?  Compare to the number needed with trapezoid in the Week07 code.

What was the smallest subinterval?  If you were to use this smallest subinterval as your panel size for a composite method, how many panels would be needed?

\item Ask WolframAlpha for ``area of batman equation'' (and use $a=1$ in the area formula that they return).  Compare this to your calculated value.  Did you actually meet the error tolerance that was specified?
\end{parts}


\question Reflection
\begin{parts}
\item When you worked on the problem set where did you get stuck or become confused?
\item What aspects of the course challenged you this week?  What did you do to address those challenges?  What topics/ideas/procedures do you not yet understand?
\item What did you understand the best this week?  What, if anything, do you understand better this week than you did in the past?
\item List the people that you worked with or consulted on the problem set problems.  This might include other students in the course, course instructors, or people who have previously taken the course.
\item Below, indicate how much of your time for this class has been doing the following activities:
	\begin{enumerate}
	\item Working on the problem set (including time in Python)
	\item Reviewing course materials, including the course textbooks
	\item Working through supplementary materials
	\item Going to office hours or lab
	\item Other (please specify)
	\end{enumerate}
\item If you used chatGPT or other AI tools, attach that information as part of this question.
\end{parts}

\end{questions}

\end{document}