\documentclass[12pt,letterpaper,noanswers]{exam}
\usepackage[usenames,dvipsnames,svgnames,table]{xcolor}
\usepackage[margin=0.9in]{geometry}
\renewcommand{\familydefault}{\sfdefault}
\usepackage{multicol}
\pagestyle{head}
\definecolor{c03}{HTML}{FFDDDD}
\header{AM 111 Class 18}{}{Skill Check}
\runningheadrule
\headrule
\usepackage{graphicx} % more modern
\usepackage{amsmath} 
\usepackage{amssymb} 
\usepackage{hyperref}
\usepackage{tcolorbox}

\begin{document}
 \pdfpageheight 11in 
  \pdfpagewidth 8.5in

\noindent Name: \rule{2.5in}{0.5pt}

\noindent Put away any notes or other materials, and work on this activity alone.

\noindent You'll receive feedback on your work and will complete a similar question on a future skill check.


\begin{questions}
\item A particular quadrature rule for $\displaystyle\int_{-1}^1f(x)dx$ is given by fitting an $n=3$ Legendre polynomial to 
$$\left(-\sqrt{\frac{3}{5}},f\left(-\sqrt{\frac{3}{5}}\right)\right), \left(0,f\left(0\right)\right), \left(\sqrt{\frac{3}{5}},f\left(\sqrt{\frac{3}{5}}\right)\right)$$
and integrating the fitted curves.

This is a composite (or adaptive) method:
\begin{oneparchoices}
\choice yes \choice no
\end{oneparchoices}

This is a Newton-Cotes method:
\begin{oneparchoices}
\choice yes \choice no
\end{oneparchoices}

How do you know?


\vspace{4.0cm}

\item (Retake for Class 15)

Find the degree of precision of the following approximation for $\displaystyle\int_{-1}^1 f(x)dx$:

$\displaystyle\int_{-1}^1 f(x)dx\approx f\left(-\frac{\sqrt{3}}{3}\right)+f\left(\frac{\sqrt{3}}{3}\right)$.

\emph{The following information may be helpful}
\[\int_{-1}^1 1\ dx = 2, \int_{-1}^1 x\ dx = 0, \int_{-1}^1 x^2\ dx = \frac{2}{3}, \int_{-1}^1 x^3\ dx =0, \int_{-1}^1 x^4\ dx = \frac{2}{5}\]

\end{questions}

\end{document}