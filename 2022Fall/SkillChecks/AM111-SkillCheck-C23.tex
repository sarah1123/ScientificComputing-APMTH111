\documentclass[12pt,letterpaper,noanswers]{exam}
\usepackage[usenames,dvipsnames,svgnames,table]{xcolor}
\usepackage[margin=0.9in]{geometry}
\usepackage{tikz}
\renewcommand{\familydefault}{\sfdefault}
\usepackage{multicol}
\pagestyle{head}
\definecolor{c03}{HTML}{FFDDDD}
\header{AM 111 Class 23}{}{Skill Check}
\runningheadrule
\headrule
\usepackage{graphicx} % more modern
\usepackage{amsmath} 
\usepackage{amssymb} 
\usepackage{hyperref}
\usepackage{tcolorbox}

\begin{document}
 \pdfpageheight 11in 
  \pdfpagewidth 8.5in

\noindent Name: \rule{2.5in}{0.5pt}

\noindent Put away any notes or other materials, and work on this activity alone.

\noindent You'll receive feedback on your work and will complete a similar question on a future skill check.


\begin{questions}
\item 
For the neural network below, how many inputs does it have?  How many units in the second hidden layer?  What is the depth of the network?

\newcommand{\inputnum}{4} 
% Hidden layer neurons'number
\newcommand{\hiddennum}{8} 
% Hidden layer neurons'number
\newcommand{\hiddennumk}{2} 
% Hidden layer neurons'number
\newcommand{\hiddennumm}{6}  
% Output layer neurons'number
\newcommand{\outputnum}{3} 
\begin{tikzpicture}
% Input Layer
\foreach \i in {1,...,\inputnum}
{
    \node[circle, 
        minimum size = 6mm,
        fill=orange!30] (Input-\i) at (0,-\i) {};
}
% Hidden Layer
\foreach \i in {1,...,\hiddennum}
{
    \node[circle, 
        minimum size = 6mm,
        fill=teal!50,
        yshift=(\hiddennum-\inputnum)*5 mm
    ] (Hidden-\i) at (2.5,-\i) {};
}
\foreach \i in {1,...,\hiddennumk}
{
    \node[circle, 
        minimum size = 6mm,
        fill=teal!50,
        yshift=(\hiddennumk-\inputnum)*5 mm
    ] (Hiddenk-\i) at (5,-\i) {};
}
\foreach \i in {1,...,\hiddennumm}
{
    \node[circle, 
        minimum size = 6mm,
        fill=teal!50,
        yshift=(\hiddennumm-\inputnum)*5 mm
    ] (Hiddenm-\i) at (7.5,-\i) {};
}
% Output Layer
\foreach \i in {1,...,\outputnum}
{
    \node[circle, 
        minimum size = 6mm,
        fill=purple!50,
        yshift=(\outputnum-\inputnum)*5 mm
    ] (Output-\i) at (10,-\i) {};
}
% Connect neurons In-Hidden
\foreach \i in {1,...,\inputnum}
{
    \foreach \j in {1,...,\hiddennum}
    {
        \draw[->, shorten >=1pt] (Input-\i) -- (Hidden-\j);   
    }
}
% Connect neurons Hidden-Out
\foreach \i in {1,...,\hiddennum}
{
    \foreach \j in {1,...,\hiddennumk}
    {
        \draw[->, shorten >=1pt] (Hidden-\i) -- (Hiddenk-\j);
    }
}
\foreach \i in {1,...,\hiddennumk}
{
    \foreach \j in {1,...,\hiddennumm}
    {
        \draw[->, shorten >=1pt] (Hiddenk-\i) -- (Hiddenm-\j);
    }
}
\foreach \i in {1,...,\hiddennumm}
{
    \foreach \j in {1,...,\outputnum}
    {
        \draw[->, shorten >=1pt] (Hiddenm-\i) -- (Output-\j);
    }
}
% Inputs
\foreach \i in {1,...,\inputnum}
{     
    \draw (Input-\i) node {$x_{\i}$};
    % \draw[<-, shorten <=1pt] (Input-\i) -- ++(-0.5,0)
    %     node[left]{$x_{\i}$};
}
% Outputs
\foreach \i in {1,...,\outputnum}
{            
\draw (Output-\i) node {$y_{\i}$};
    % \draw[->, shorten <=1pt] (Output-\i) -- ++(0.5,0)
    %     node[right]{$y_{\i}$};
}
\end{tikzpicture}

\vspace{2cm}

\item (Retake for Class 20)

Use Taylor expansion to first order to write $f(t_k+h/2,y_k+hf(t_k,y_k)/2)$ in terms of $f(t_k, y_k)$,$\dfrac{\partial f}{\partial t}(t_k,y_k)$, $\dfrac{\partial f}{\partial y}(t_k,y_k)$, and $\mathcal{O}(h^2)$.

\vspace{3cm}

\item (Retake for Class 21)

There are $s = 132300$ data points in a data set with $6$ seconds of data.  Find the frequency associated with the $24$rd Fourier mode ($c_{24}$) in units of $1/$seconds (Hertz).

\end{questions}

\end{document}
