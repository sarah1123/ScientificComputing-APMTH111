\documentclass[12pt,letterpaper,noanswers]{exam}
\usepackage[usenames,dvipsnames,svgnames,table]{xcolor}
\usepackage[margin=0.9in]{geometry}
\renewcommand{\familydefault}{\sfdefault}
\usepackage{multicol}
\pagestyle{head}
\definecolor{c03}{HTML}{FFDDDD}
\header{AM 111 Class 08}{}{Skill Check}
\runningheadrule
\headrule
\usepackage{graphicx} % more modern
\usepackage{amsmath} 
\usepackage{amssymb} 
\usepackage{hyperref}
\usepackage{tcolorbox}

\begin{document}
 \pdfpageheight 11in 
  \pdfpagewidth 8.5in

\noindent Name: \rule{2.5in}{0.5pt}

\noindent Put away any notes or other materials, and work on this activity alone.

\noindent You'll receive feedback on your work and will complete a similar question on a future skill check.


\begin{questions}
\item For sufficiently large $k$, $\dfrac{\vert e_{k+1}\vert}{\vert e_k\vert^r} \approx C$


$C$ and $r$ are estimated from successive steps of a numerical algorithm below.

\begin{verbatim}
9.772653001316197  -0.7037877659013353
0.278488253844298 1.2819516332680758
0.186632524973088 1.6748892519066119
0.185207756022360 1.9548039422643921
0.199087189401616 1.9990631687120175
-0.19999739183188 1.9999982769025637
\end{verbatim}

Assuming a simple root, is this sequence generated by the bisection method, Newton's method, or the secant method?

\emph{These sequences were generated via the Python code that we used in class during Class07}

\vspace{4cm}

\item (Retake for Class 05)
Given $A\vc{w} = \vc{y}$, with the system overdetermined, construct the normal equations for this system.

Let $A = \left[\begin{array}{r r}
1 & -1 \\
0 & 2 \\
2 & 0
\end{array}\right]$ and $\vc{y} =  \left[\begin{array}{r}
6 \\ 1 \\ 4 \end{array}\right]$.  

\emph{You do not need to complete the matrix multiplications.}

\end{questions}

\end{document}