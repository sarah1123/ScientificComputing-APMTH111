\documentclass[12pt,letterpaper]{exam}
%\usepackage{color}
\usepackage[usenames,dvipsnames,svgnames,table]{xcolor}
\usepackage[margin=0.7in]{geometry}
\renewcommand{\familydefault}{\sfdefault}
\usepackage{multicol}
\pagestyle{head}
\definecolor{c01}{HTML}{FFBBDD}
\header{AM 111 Notes}{Intro to Scientific Computing - F22}{\today.}{}
\runningheadrule
\headrule
\usepackage{diagbox}
\usepackage{graphicx} 
\usepackage{amsmath} 
\usepackage{amssymb} 
\usepackage{hyperref}
\usepackage{enumitem}
\usepackage[
backend=biber,
style=alphabetic,
sorting=ynt
]{biblatex}
\addbibresource{syllabus.bib}

%\setlist{nosep} % or \setlist{noitemsep} to leave space around whole list
\usepackage{sectsty}  % Make section fonts 12 point
% make a task list:
\newlist{todolist}{itemize}{2}
\setlist[todolist]{label=$\square$}

% Adds a vertical column separator.
\setlength{\columnseprule}{1pt}
\def\columnseprulecolor{\color{black}}

\sectionfont{\fontsize{12}{15}\selectfont}
\usepackage{tcolorbox} % Nice boxes

\begin{document}
\pdfpageheight 11in 
\pdfpagewidth 8.5in

\section{Planning}
\begin{enumerate}
\itemsep0pt
    \item outline for a course based on Petros' course
\end{enumerate}

\section{Resources}
\subsection{People I might contact}
Python support: Tim, Eli, Petros, active learning labs


\subsection{Textbooks}

Dina used Numerical Methods in Engineering with Python3 \url{https://www-cambridge-org.ezp-prod1.hul.harvard.edu/core/books/numerical-methods-in-engineering-with-python-3/95151C37C2F427F30DC90FA619FE79F9}

Cengiz used Greenberg and Chartier

Petros has a list of 4 (grab those)

Chris uses Scientific Computing: An Introductory Survey, by Michael T. Heath for AM205

\section{Content}
\subsection{}

\section{AM 205}
\subsection{Topics/Problems}
\begin{enumerate}
\itemsep0pt
    \item intro + error (continuous processes become discrete and computers are finite precision so need to round) + idea of a condition number (change in output over output divided by change in input over input: how sensitive is the output to small changes in the input) + IEEE floating point standards
\end{enumerate}

\subsection{Ed notes}
PSet 01: Q1. Lagrangian polynomail; Vandermonde matrix and inversion.  Is the Vandermonde illconditioned (lecture notes say these are a bad idea).  Is there a threshold for the condition number? TF says: Error rule of thumb: one decimal digit of accuracy in $x$ (for $Ax = b$ for each power of $10$ in the condition number of $A$.   \url{http://www.ece.northwestern.edu/local-apps/matlabhelp/toolbox/control/numerical/relcomp2.html}

PSet 01 Q2: Chebyshev vs other cubic polynomials (find something lower error than the Chebyshev)



\end{document}
