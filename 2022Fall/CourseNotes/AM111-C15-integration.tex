\documentclass[12pt,letterpaper,noanswers]{exam}
\usepackage[usenames,dvipsnames,svgnames,table]{xcolor}
\usepackage[margin=0.9in]{geometry}
\renewcommand{\familydefault}{\sfdefault}
\usepackage{multicol}
\pagestyle{head}
\header{AM 111 Class 15}{}{Approximating integrals, p.\thepage}
\runningheadrule
\headrule
\usepackage{siunitx}
\usepackage{graphicx} % more modern
\usepackage{amsmath} 
\usepackage{amssymb} 
\usepackage{hyperref}
\usepackage{tcolorbox}
\usepackage{enumitem}
\def\mbf{\mathbf}
\newcommand{\vc}[1]{\boldsymbol{#1}}
\def\dsst{\displaystyle}
\DeclareMathOperator*{\argmin}{arg\,min} % thin space, limits underneath in displays


\begin{document}
 \pdfpageheight 11in 
  \pdfpagewidth 8.5in

\noindent 

\section*{Preliminaries}

\begin{itemize}
\itemsep0pt
\item Problem set 06 is due by Friday at 5pm.  Submit extension requests in advance of that time.
\item There is a skill check in the next class.
\item Monday and Wednesday office hours will be held on Zoom next week.  Thursday 1-2pm and 4-6pm office hours will be held on zoom.  (Tuesday 2:30-3:30pm and Thursday 2-3pm will be held in Pierce G11).
\end{itemize}


\noindent\textbf{Big picture}

Today: Approximating $\int_{a}^{b}f(x)dx$.

\vspace{0.2cm}
\hrule
\vspace{0.2cm}

\noindent \textbf{Skill check practice}
\begin{questions}
\item The error from using the trapezoid rule to approximate $\int_a^b f(x)dx$ with a single panel is given by $-\frac{1}{12}h^3f''(c_1)$, where $h$ is the panel size.  What is the error for the composite rule?

% \item A quadrature rule can be written in the form $\sum\limits_{k=1}^n a_k f(x_k)$.

% For the composite midpoint rule on two panels, given by $(b-a)\frac{1}{2}\left(f(a + \frac{b-a}{4})+ f(a + 3\frac{b-a}{4})\right)$, identify $n$ and $a_k, x_k$ for $k=1,...,n$

%Trapezoid rule: $(b-a)\frac{1}{2}(f(a)+f(b)$
%Simpson's rule: 
\item The skill from the Class 11 handout (Skill Check C12).
\end{questions}


\vspace{0.2cm}
\hrule
\vspace{0.2cm}

\noindent \textbf{Skill check solution}
\begin{questions}
\item $-\frac{1}{12}(b-a)h^2f''(c)$.  

Explanation:
For $m$ panels evenly splitting $[a,b]$, this comes from $\sum\limits_{k=1}^m -\frac{1}{12}h^3f''(c_k) = -\frac{1}{12}mh^3f''(c) = -\frac{1}{12}(b-a)h^2f''(c)$ (from the generalized intermediate value theorem and noting $mh = (b-a)$).

In general for error $\alpha h^nf''(c_1)$ on a single panel, the error will be $\alpha (b-a)h^{n-1}f''(c)$ for the composite rule.
% \item There are two points in the calculation so $n = 2$.

% $a_k = \frac{1}{2}(b-a)$ for both points.

% $x_1 = a+(b-a)/4$

% $x_2 = a+3(b-a)/4$.
\item See the past handout.
\end{questions}
\vspace{0.2cm}
\hrule
\vspace{0.2cm}

\noindent \textbf{Teams}
\begin{multicols}{3}
1. Mina, Basil, Johan

2. Nini, Brian, Eli

3. Nicolas, Aidan, Julia K

4. Mack, Benjamin, Robert

5. Alex, RJ, Jessica

6. Caitlin, Nina, Daniyal

7. Cameron, Dani, Emma

8. Eletria, Julia M, Tom

9. Ray, Ivonne, Shang

10.  Sophie, Eric, Alex

11. Jack, Esmé, Zachary

12. Kevin, Kevin, Marissa

\end{multicols}


\section*{Numerical Integration}
\subsection*{Newton-Cotes formulas}
Many integrals cannot be evaluated analytically.


\begin{tcolorbox}
\begin{itemize}
\itemsep0pt
    \item The trapezoid rule is a closed Newton-Cotes formula: $\int_a^b f(x)dx \approx \frac{1}{2}(f(a)+f(b))(b-a)$.  
    \item The {\bf degree of precision} of a numerical integration method is the greatest integer $k$ for which all polynomials of degree $k$ or less are integrated exactly by the method.

    \end{itemize}
    \end{tcolorbox}
    \begin{tcolorbox}
    \begin{itemize}
    \itemsep0pt
        \item The trapezoid rule has a degree of precision is $1$: ($p(x) = c_1 x + c_0$ will be integrated exactly by the rule).
    \item The error from using the trapezoid rule to approximate $\int_a^b f(x)dx$ is $-\frac{1}{12}(b-a)^3f''(c)$ where $c$ is a point in $[a,b]$.
\end{itemize}
\end{tcolorbox}






%\noindent\textbf{Simpson's rule}
\begin{tcolorbox}
\textbf{Simpson's rule} is generated by using quadratic interpolation on $f$ with interpolation points $x_0 = a, x_1 = \frac{a+b}{2}, x_2 = b$.
\end{tcolorbox}
\begin{enumerate}[resume=classQ]
    \item Let $p_2(x) = y_0L_1(x) + y_1L_2(x)+y_2L_3(x)$ where $x_0 = a, x_1 = (a+b)/2, x_2 = b$ and $y_k = f(x_k)$.  
    
    $\displaystyle\int_a^b p_2(x) = A_0 y_0 + A_1 y_1 + A_2 y_2$.  You can find $A_0, A_1, A_2$ by integrating $L_1, L_2, L_3$.  Instead of integrating, use the integral information in the caption of the figure below to find $A_0, A_1, A_2$.  Note that $h = (b-a)/2$.
\end{enumerate}


    \includegraphics{img/Class12Sauerintegrals.png}
    
    
\vspace{0.5in}

% \begin{enumerate}[resume=classQ]
%     \item The \textbf{method of undetermined coefficients} is another option for finding $A_0, A_1, A_2$.
%     \begin{parts}
%     \item For $f(x)$ a constant, linear, or quadratic polynomial, $\int_a^b f(x)dx = A_0 y_0 + A_1 y_1 + A_2 y_2$ is exact.
    
%     Let $f(x) = 1$.  Use $\int_a^b 1\ dx = A_0  + A_1 + A_2 $ to find an equation for $A_0+A_1+A_2$ in terms of $a$ and $b$.
%     \vspace{1in}
%     \item Repeat this for $f(x) = x$ and $f(x) = x^2$ to generate two more equations in terms of $A_0, A_1, A_2$ and $a,b$.
%     \vspace{2in}
%     \item Write your linear system as a matrix equation.
%     \vfill
%     \end{parts}
    
% The solution to the system is $A_0 = A_2 = \dfrac{b-a}{6}$, $A_1 = 4\dfrac{b-a}{6}$.
% \end{enumerate}
\begin{tcolorbox}
\noindent\textbf{Simpson's rule}

Given a function $f(x) \in C^4[a,b]$, let $h = \dfrac{b-a}{2}$.
\[
          \int_{a}^{b} f(x) \, dx = \frac{h}{3}\left[ f(a)+4f\left(\dfrac{a+b}{2}\right) + f(b) \right]
          - \frac{h^5f^{(iv)}(c)}{90}.\]
\end{tcolorbox}



\begin{enumerate}[resume=classQ]
    \item Show that the degree of precision for Simpson's rule is $3$.
    
    To do this, we need to show that $\int_a^b 1\ dx$, $\int_a^b x\ dx$, $\int_a^b x^2\ dx$, $\int_a^b x^3\ dx$ are all integrated exactly by Simpson's rule.

$\int_a^b 1\ dx$, $\int_a^b x\ dx$, $\int_a^b x^2\ dx$ are integrated exactly.  Show that $\int_a^b x^3\ dx$ is as well.
    \begin{parts}
    \item One way to do this is to shift your integral.  Find $k$ such that $\int_a^b x^3 dx = \int_{-h}^h (x+k)^3 dx$.
    \vspace{1cm}
    
    \item Expanding, $(x+k)^3 = x^3 + q(x)$ where $q(x)$ is a quadratic function.  
    
    $\int_a^b x^3 dx = \int_{-h}^h (x+k)^3 dx = \int_{-h}^h x^3 dx  + \int_{-h}^h q(x) dx$
    
    Provide an argument for why each term of the final expression is zero.
    \vspace{0.8in}
    
    
    
    \end{parts}
    
\end{enumerate}

\subsection*{Composite rules (from Sauer \S 5.2.3)}
\begin{tcolorbox}

\begin{itemize}
\itemsep0pt
    \item For a composite rule for $\displaystyle\int_a^b f(x)dx$, break $[a,b]$ into adjacent subintervals, or \textbf{panels}.
    \item The \textbf{composite Trapezoid rule} is the sum of the Trapezoid rule applied to each panel.
    \item For a function $f\in C^2[a,b]$, choose an evenly spaced grid: $a = x_0<x_1<...<x_{m-1}<x_m = b$ with $h = x_k-x_{k-1} = (b-a)/m$.  \[\displaystyle\int_a^b f(x)dx = \frac{h}{2}\left(y_0 + \sum\limits_{k=2}^{m-1}y_k + y_m \right) - (b-a)\frac{h^2}{12}f''(c),\] where $c$ is between $a$ and $b$.
\end{itemize}
\end{tcolorbox}
\begin{enumerate}[resume=classQ]
    \item In a single panel, the error from the Trapezoid rule is given by $-\frac{1}{12}h^3f''(c_k)$ with $x_{k-1}<c_k<x_k$.
    \begin{parts}
    \item Find an expression for the sum of the error over all $m$ panels (use summation notation).
    
\vspace{0.6in}
\item The generalized intermediate value theorem says that for $f$ continuous on $[a,b]$, $c_k$ points in $[a,b]$, and weights $w_1, ..., w_m>0$, there exists a number $c$ between $a$ and $b$ such that $w_1f(c_1)+...+w_mf(c_m) = (w_1+...+w_m)f(c)$.  Use this theorem to write your expression as $f''(c)$ times a constant.  Write the constant in terms of $m$ and $h$.
\vspace{0.7in}
\item Is the error you found equal to $- (b-a)\frac{h^2}{12}f''(c)$?
%\vspace{1cm}
    \end{parts}
\end{enumerate}

The error in the Trapezoid rule is proportional to $f''$ and to $h^3$.

The error in the composite Trapezoid rule is proportional to $f''$ and to $h^2$.

\begin{enumerate}[resume=classQ]
\item For composite Simpson's rule, let each panel be length $2h$.  Use the grid $a = x_0 < x_1 < ... < x_{2m-1}<x_{2m}$.
\begin{parts}
\item For $a = x_0, b = x_2$, Simpson's rule can be written $\int_a^b f(x)dx \approx \frac{h}{3}(y_0 + 4y_1 + y_2)$.

The composite rule is of the form \[\frac{h}{3}\left[y_0 + A \sum\limits_{k=1}^m y_{2k-1} + B\sum\limits_{k=1}^{m-1}y_{2k} + y_{2m}\right].\]

Find $A$ and $B$.
\vspace{0.7in}
\end{parts}
\end{enumerate}

\subsection*{Open vs closed formulas}
\begin{tcolorbox}
\textbf{Closed} Newton-Cotes formulas use $f(a)$ and $f(b)$ in the approximation.

\textbf{Open} Newton-Cotes formulas avoid values from the ends of the intervals.
\end{tcolorbox}
\begin{enumerate}[resume=classQ]
    \item To approximate $\displaystyle\int_a^b f(x)dx$, let $x_0 = a, x_1 = b$, $h = b-a$.  Set $w = x_0 + h/2$ (the midpoint of the interval).
    \begin{parts}
    \item Find the degree $1$ Taylor expansion of $f(x)$ about the midpoint $w$.  Do this exactly, so include the remainder term (it is $\frac{1}{2}(x-w)^2f''(c_x)$.
    \vspace{0.7in}
    \item Integrate both sides from $a$ to $b$.
    
    Note that $f'(w)$ is a constant.  Use the mean value theorem for integrals to remove $f''(c_x)$ from the integral.
    \vspace{0.7in}
    
    \end{parts}
This generates the \textbf{midpoint rule} for approximating an integral.
\end{enumerate}
\begin{tcolorbox}
The \textbf{composite midpoint rule} is given by \[\int_a^bf(x)dx = h\sum\limits_{i=1}^m f(w_i) + \frac{1}{24}(b-a)h^2 f''(c)\] where $h = (b-a)/m$ and $a<c<b$.  The $w_i$ are midpoints of the $m$ equal subintervals of $[a,b]$.
\end{tcolorbox}
% \fbox{
%     \begin{minipage}{15cm}
%       \subsection*{Midpoint Rule}
%       Given a function $f(x) \in C^2(x_0,x_1)$ and points $x_0<x_1$, $x_1-x_0 = h$,
%             \begin{equation}
%           \int_{x_0}^{x_1} f(x) \, dx = h(f(x_0+ h/2)) + \frac{h^3}{24} f''(c)
%         \end{equation}
%     \end{minipage}}

% \vspace{10cm}


%     \fbox{
%     \begin{minipage}{15cm}
%       \subsection*{Composite Midpoint Rule}
%       Given a function $f(x) \in C^4(a,b)$ and equally spaced
%       points $a=x_0<x_1<\cdots<x_{2m}=b$, $x_{i+1}-x_i = h$,
%       \begin{equation}
%         \int_a^b f(x) \, dx = h \sum_{i=1}^m f\left(\frac{x_{i-1} + x_i}{2}\right) + \frac{(b-a)h^2f''(\xi)}{24}.
%       \end{equation}
%     \end{minipage}}

% \pagebreak

% \subsection*{Adaptive Quadrature}

% \pagebreak

% \been

% \item Read the adaptive quadrature algorithm in the book (\S5.4) and implement it to estimate:

% \beit
% \item $\int_{-1}^1   2\sqrt{1-x^2} dx$
% \item $\int_{-1}^1   1+\sin(e^{3x}) dx$
% \enit

% \enen



%
%\subsection*{Romberg Integration}
%
%\been
%
%\item[4.] Romberg integration uses the composite trapezoid rule with successively
%smaller stepsizes $h$.
%
%\been
%\item {\bf Romberg Integration (First Column)}
%
%Define stepsizes $h_j$ such that each successive stepsize is half the
%previous stepsize:
%\begin{align*}
%h_1 &= b-a \\
%h_2 &= (b-a)/2 \\ 
%h_3 &= (b-a)/4 \\
%& \vdots \\
%h_j&=\frac{1}{2^{j-1}}(b-a).
%\end{align*}
%
%The first column of Romberg integration, $R_{j1}$, is the composite
%trapezoid rule applied to the function on $[a,b]$ with stepsize $h_j$.  Write down the formula for the first few elements of the first column:
%
%\vfill
%
%\begin{align*}
%R_{11} &= \blank{3in} \\
%& \\
%R_{21} &= \blank{3in} \\
%& \\
%R_{31} &= \blank{3in}
%\end{align*}

%\pagebreak
%
%The second column applies Richardson extrapolation to the 1st column
%$R_{j1}$.  Write down the results for the second column.  What is $R_{22}$?
%
%\vfill
%
%\begin{align*}
%R_{22} &= \frac{2^2 R_{21} - R_{11}}{3} = \blank{3in} \\
%& \\
%R_{32} &= \frac{2^2 R_{31} - R_{21}}{3} = \blank{3in}
%\end{align*}
%
%\item Calculate $R_{11}$, $R_{21}$, and $R_{22}$ for $\dsst{\int_0^4 x^2 \, dx}$.
%
%\vfill
%
%\pagebreak
%
%\item The third column applies Richardson extrapolation to the second column.
%Because the second column is composite Simpson's rule, it is 4th order in $h$:
%\begin{align*}
%R_{33} &= \frac{4^2 R_{32} - R_{22}}{4^2-1} \\
%R_{43} &= \frac{4^2 R_{42} - R_{32}}{4^2-1} \\
%& \vdots \\
%R_{j3} &= \frac{4^2 R_{j2} - R_{j-1,2}}{4^2-1}
%\end{align*}
%
%\item This pattern perpetuates
%$$
%R_{jk} = \frac{4^{k-1} R_{j,k-1} - R_{j-1,k-1}}{4^{k-1}-1} 
%$$
%
%\enen
%
%\item[5.] Computer Examples
%\been
%
%\item Estimate $\dsst{\int_{-1}^1 2\sqrt{1-x^2} \, dx}$.
%
%\vspace{1cm}
%
%\item Estimate $\dsst{\int_{-1}^1 1 + \sin{e^{3x}} \, dx}$.
%
%\vspace{1cm}
%
%\enen
%
%\enen


\end{document}