\documentclass[12pt,letterpaper,noanswers]{exam}
%\usepackage{color}
\usepackage[usenames,dvipsnames,svgnames,table]{xcolor}
\usepackage[margin=0.9in]{geometry}
\renewcommand{\familydefault}{\sfdefault}
\usepackage{multicol}
\pagestyle{head}
\definecolor{c02}{HTML}{FFBBBB}
\definecolor{c03}{HTML}{FFDDDD}
\header{AM 111 First Reflection}{}{} %Updated on \today.}{}
\runningheadrule
\headrule
\usepackage{diagbox}
\usepackage{graphicx} % more modern
%\usepackage{subfigure} 
\usepackage{amsmath} 
\usepackage{amssymb} 
%\usepackage{gensymb} 
%\usepackage{natbib}
\usepackage{hyperref}
%\usepackage{enumitem}
%\setlength{\parindent}{0pt}
%\usepackage{setspace}
%\pagestyle{empty}  
%\newcommand{\Sc}[0]{
%{\color{BlueViolet}\S}
%}
\usepackage{tcolorbox}
\usepackage[framed,numbered,autolinebreaks,useliterate]{mcode}

\begin{document}
 \pdfpageheight 11in 
  \pdfpagewidth 8.5in
\noindent  Name:

\begin{enumerate}
    \item Why are you taking this course? What are your goals for yourself?
    \vspace{3.5in}
    
\item Articulate a plan for how you will “do” this course. In your plan, please include how much time you will allocate to this course each week, what you will do when you encounter challenges, and so forth. Be as specific and granular as you can, referencing the components of the course.
\vspace{3in}
\eject
\item Keeping in mind the goals you articulated above, what specific criteria / metrics / experiences / reflections will you use to evaluate yourself? In other words, how will you know if you are meeting your goals? \emph{I am not asking you to evaluate yourself yet. I am asking you to come up with a specific plan for how you will evaluate yourself in the future.}
\vspace{2in}

%\item Now, using your response to the previous question, please evaluate your work in this course so far.
%\vspace{3in}
% \item You'll be revising your answers to this self-evaluation form regularly. In light of your own evaluation of yourself, is there anything in your approach to this course that you will plan to change between now and the next time you fill out this form? If so, please explain.
\end{enumerate}


\end{document}
