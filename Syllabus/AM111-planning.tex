\documentclass[12pt,letterpaper]{article}

\usepackage[margin=0.9in]{geometry}
\renewcommand{\familydefault}{\sfdefault}
\usepackage{fancyhdr}
\pagestyle{fancy}

\lhead{AM 111 Planning}
\chead{Updated on \today.}
\rhead{{\color{ForestGreen} Fall 2022}}

\usepackage{color}
\usepackage{amsmath,amssymb}
\usepackage{hyperref}
\usepackage[usenames,dvipsnames,svgnames,table]{xcolor}
\usepackage{termcal}

%\usepackage{color}
%\setlength{\belowcaptionskip}{-10pt}
%\setlength{\headheight}{15.2pt}
%\setlength{\parindent}{.5 in}
%\evensidemargin -0.23in
% \oddsidemargin -0.23in
% \setlength\textheight{10in}
% \setlength\textwidth{6.75in}
% %\setlength\columnsep{0.25in}
% \setlength\headheight{0pt}
% \setlength\headsep{0pt}
% \addtolength{\topmargin}{-60pt}

\definecolor{c00}{HTML}{FFBBFF}
\definecolor{c01}{HTML}{FFBBDD}
\definecolor{c02}{HTML}{FFBBBB}
\definecolor{c03}{HTML}{FFDDDD}
\definecolor{c04}{HTML}{FFDDBB}
%\definecolor{c04}{HTML}{DDFFDD}
\definecolor{c05}{HTML}{FFFFBB}
%\definecolor{c04}{HTML}{FFFFBB}
\definecolor{c06}{HTML}{DDFFBB}
\definecolor{c07}{HTML}{BBFFBB}
\definecolor{c08}{HTML}{BBFFFF}
\definecolor{c09}{HTML}{BBDDFF}
\definecolor{c10}{HTML}{BBBBFF}
\definecolor{c11}{HTML}{DDBBFF}
\definecolor{c12}{HTML}{BBBBDD}


\newcommand{\claa}[1] {\colorbox{c00}{\makebox[2.3cm][l]{#1} } \\}
\newcommand{\cla}[1] {\colorbox{c01}{\makebox[2.3cm][l]{#1} } \\}
\newcommand{\clb}[1] {\colorbox{c02}{\makebox[2.3cm][l]{#1} } \\}
\newcommand{\clc}[1] {\colorbox{c03}{\makebox[2.3cm][l]{#1} } \\}
\newcommand{\cld}[1] {\colorbox{c04}{\makebox[2.3cm][l]{#1} } \\}
\newcommand{\cle}[1] {\colorbox{c05}{\makebox[2.3cm][l]{#1} } \\}
\newcommand{\clf}[1] {\colorbox{c06}{\makebox[2.3cm][l]{#1} } \\}
\newcommand{\clg}[1] {\colorbox{c07}{\makebox[2.3cm][l]{#1} } \\}
\newcommand{\clh}[1] {\colorbox{c08}{\makebox[2.3cm][l]{#1} } \\}
\newcommand{\cli}[1] {\colorbox{c09}{\makebox[2.3cm][l]{#1} } \\}
\newcommand{\clj}[1] {\colorbox{c10}{\makebox[2.3cm][l]{#1} } \\}
\newcommand{\clk}[1] {\colorbox{c11}{\makebox[2.3cm][l]{#1} } \\}
\newcommand{\cll}[1] {\colorbox{c12}{\makebox[2.3cm][l]{#1} } \\}
%\newcommand{\exam}[1] {\colorbox{c03}{\makebox[0.8cm][l]{#1}} \colorbox{c04}{\makebox[0.8cm][l]{ }} \colorbox{c05}{\makebox[0.8cm][l]{}} \colorbox{c06}{\makebox[0.8cm][l]{} } \\}

\newcommand{\TuThClass}{\skipday
\calday[Tuesday]{\classday}
\skipday % Wednesday
\calday[Thursday]{\classday}
\skipday %
\skipday\skipday % weekend (no class)
}
\newcommand{\MWClass}{% don't start on a skipday.
\calday[Monday]{\classday}
\skipday
%\calday[Tuesday]{\noclassday}
\calday[Wednesday]{\classday}
\skipday%\calday[Thursday]{\noclassday}
\calday[Friday]{\classday}
\skipday
\skipday}

\newcommand{\calnext}[2]
{\caltextnext{\textbf{#1} #2}}

\newcommand{\Holiday}[2]{%
\options{#1}{\noclassday}
\caltext{#1}{#2}
}


\begin{document}

\section{Timeline}
\subsection{Project brief}

\noindent Goals:
\begin{itemize}
\itemsep0pt
    \item internal: examples of interest to engineering and applied math students
    \item internal: student learning of using computers to approach math problems and limits/challenges
    \item external: materials others can use for a similar course
\end{itemize}

\noindent Stakeholders:
\begin{itemize}
\itemsep0pt
    \item students
    \item course staff
    \item other faculty in AM/SEAS
    \item faculty at other universities
\end{itemize}

\noindent Timeframe:

\begin{itemize}
\itemsep0pt
    \item between August 16th and December 16th, every part of this project will happen.
\end{itemize}

\noindent Key milestones:
\begin{itemize}
\itemsep0pt
    \item syllabus
    \item progress assessments
    \item final assignment
\end{itemize}


\noindent Due dates:
\begin{enumerate}
    \item Aug 16: syllabus and basic Canvas site \href{https://teaching.berkeley.edu/sites/default/files/syllabus_components.pdf}{(Syllabus parts reference - Berkeley)}
    \begin{enumerate}
        \item topic list and calendar (both subject to change)
        \item list of all assignments (pre-class?  or just in-class?)
        \item dates for assessments
        \item grading information
        \item learning objectives
        \item overview of content of the course
    \end{enumerate}
    \item Sept 1: full Canvas site, PSet 01, classes 01-03
    \item Weekly
\end{enumerate}

Timeline:


\section{Other courses}
\subsection{AM205}
\href{https://people.math.wisc.edu/~chr/am205/}{AM205 website}

After taking this course, students should be able to:
\begin{enumerate}
\itemsep0pt
    \item Apply standard techniques to analyze key properties of numerical algorithms, such as stability and convergence.
    \item Understand and analyze common pitfalls in numerical computing such as ill-conditioning and instability.
    \item Perform data analysis efficiently and accurately using data fitting methods.
    \item Derive and analyze numerical methods for ODEs and PDEs.
\item Perform optimization using well-established algorithms.
\item Implement a range of numerical algorithms efficiently in a modern scientific computing programming language.
\end{enumerate}

\section{Topics}
\subsection{floating point}
\begin{itemize}
\itemsep0pt
    \item condition number vs stability of an algorithm
    \item continuous vs discrete and number representation in computers
\end{itemize}

\subsection{linear least squares}
\subsection{root finding}
\subsection{interpolation}

\subsection{differentiation}
\subsection{integration}
\subsection{ODEs}
\subsection{fourier}
\subsection{neural nets}

\section{Calendar}

%Topics and meeting dates.

%A few to-do items:  

%Surface past math knowledge (mention we'll need x topic.  Give an example.  Give an exercise or two for them to work on.)

%Find an initial state for all students in the course using Susan Ambrose's assessment...

\begin{center}
\begin{calendar}{8/29/2022}{15} % Class starts on Thurs Sept 1st and lasts for 
% 13 weeks + finals weeks
\setlength{\calboxdepth}{.3in}
\setlength{\calwidth}{6.75in}
\TuThClass


\caltexton{1}{Modeling data: linear least squares}
\calnext{}{Modeling data: More least squares}
\calnext{}{Nonlinear systems: root finding with secant and bisection}
\calnext{}{Nonlinear systems: Newton's method}
% video assignment 
%\caltext{9/11/2016}{\S 12.5 Functions of three variables}
\calnext{}{Nonlinear systems: Root finding in systems}
\calnext{}{Nonlinear systems: Newton's method in systems}
\calnext{}{Interpolation: Lagrange interpolation}
\calnext{}{Interpolation: Cubic splines}
\calnext{}{Integration: rectangle and trapezoidal}
\calnext{}{Integration: Simpson's}
\calnext{}{Integration: Quadrature 2}
\calnext{}{Integration: Quadrature 3} 
\calnext{}{Integration: Adaptive quadrature}
\calnext{}{Quiz}
\calnext{}{Integration: Adaptive quadrature}
\calnext{}{Integration: Probability 1}
\calnext{}{Integration: Probability 2}
\calnext{}{Integration: Monte Carlo 1}
\calnext{}{Integration: Monte Carlo 2}
\calnext{}{Neural nets (NN): single layer 1}
\calnext{}{NN: single layer 2}
\calnext{}{NN: PCA}
\calnext{}{NN: autoencoders}
\calnext{}{NN: deep learning 1}
\calnext{}{NN: deep learning 2}
\calnext{}{Quiz}

% Reading week
\Holiday{5/8/2019}{\color{ForestGreen} Reading period}
\Holiday{5/6/2019}{\color{ForestGreen} Reading period}
\Holiday{5/10/2019}{\color{ForestGreen} Final Exams.  Final presentation on Saturday May 11th, 2-5pm}
\Holiday{5/3/2019}{\color{ForestGreen} Reading period}
\Holiday{2/1/2019}{\color{ForestGreen} No class}
\Holiday{2/15/2019}{\color{ForestGreen} No class}
\Holiday{2/18/2019}{\color{ForestGreen} University Holiday}
\Holiday{3/1/2019}{\color{ForestGreen} No Class (Practice Quiz)}
\Holiday{3/15/2019}{\color{ForestGreen} No class}
\Holiday{3/18/2019}{\color{ForestGreen} Spring Break}
\Holiday{3/20/2019}{\color{ForestGreen} Spring Break}
\Holiday{3/22/2019}{\color{ForestGreen} Spring Break}
\Holiday{3/29/2019}{\color{ForestGreen} No Class (Practice Quiz)}
\Holiday{4/19/2019}{\color{ForestGreen} No class}
%\Holiday{3/16/2016}{\color{ForestGreen} Spring Break}
% ... and so on

\caltext{2/1/2019}{{\color{blue} Study cards due}}
\caltext{4/17/2019}{{\color{blue} Evening Exam}}
%\caltext{2/18/2019}{\color{blue} University Holiday}
%\caltext{3/18/2019}{\color{blue} University Holiday}
%\caltext{3/20/2019}{\color{blue} University Holiday}

\options{8/30/2022}{\noclassday} 
\options{9/5/2022}{\noclassday}
%\options{10/10/2022}{\noclassday} 
\options{11/23/2022}{\noclassday}
\options{11/24/2022}{\noclassday}
\options{11/25/2022}{\noclassday}
\options{12/6/2022}{\noclassday}
\options{12/8/2022}{\noclassday}


\caltext{9/6/2022}{\color{blue} Last day to add}
\caltext{9/29/2022}{\color{blue} Last class before drop}
\caltext{11/22/2022}{\color{blue} Class may be held online}
\caltext{11/24/2022}{\color{blue} Thanksgiving (no class)}
\caltext{12/6/2022}{\color{blue} Reading period}
\caltext{12/8/2022}{\color{blue} Exams begin}


%
\caltext{2/21/2022}{\color{blue} Presidents' day}
\options{3/4/2022}{\noclassday} %
\caltext{3/4/2022}{\color{blue} No class}
% Friday before President's Day
% Friday before spring break
\options{3/11/2022}{\noclassday} %
\caltext{3/11/2022}{\color{blue} No class}
\options{3/14/2022}{\noclassday} %
\caltext{3/14/2022}{\color{blue} Spring break}
\options{3/16/2022}{\noclassday} %
\caltext{3/16/2022}{\color{blue} Spring break}
\options{3/18/2022}{\noclassday} %
\caltext{3/18/2022}{\color{blue} Spring break}



\end{calendar}
\end{center}

\eject
\noindent\textbf{Calendar of topics:} (subject to adjustment)

\begin{tabular}{| l | l | p{10cm} | l  |}
\hline
Date & Due &  Class & Text \\
\hline
\hline
   Thurs 09/01 & &  C01: Base 10, binary, numbers in computers &  \\
     \hline
   Tues 09/06  & &  C02: linear least squares & \\
  Thurs 09/08 & & C03: & \\
  \hline
    Tues 09/13  & &  C04: & \\
  Thurs 09/15 & & C05: & \\
  \hline
     Tues 09/20  & &  C06: & \\
  Thurs 09/22 & & C07: & \\
  \hline
     Tues 09/27  & &  C08: & \\
  Thurs 09/29 & & C09: & \\
  \hline
     Tues 10/04  & &  C10: & \\
  Thurs 10/06 & & C11: & \\
  \hline
     Tues 10/11  & &  C12: & \\
  Thurs 10/13 & & C13: & \\
  \hline
     Tues 10/18  & &  C14: & \\
  Thurs 10/20 & & C15: & \\
  \hline
     Tues 10/25  & &  C16: & \\
  Thurs 10/27 & & C17: & \\
  \hline
     Tues 11/01  & &  C18: & \\
  Thurs 11/03 & & C19: & \\
  \hline
     Tues 11/08  & &  C20: & \\
  Thurs 11/10 & & C21: & \\
  \hline
     Tues 11/15  & &  C22: & \\
  Thurs 11/17 & & C23: & \\
  \hline
     Tues 11/22  & &  C24: & \\
  \hline
     Tues 11/29  & &  C25: & \\
  Thurs 12/01 & & C26: & \\
  \hline
 
\hline
Mon 04/27 &  & C36: more diff eqs & \\
Wed 04/29 &  & C37: semester review & \\
Fri 05/01 & PSet 10 & & no class \\
\hline
May 05/08 &  & Final exam from 2-5pm & \\
\hline
\end{tabular}
\end{document}
