\documentclass[12pt,letterpaper,noanswers]{exam}
\usepackage[usenames,dvipsnames,svgnames,table]{xcolor}
\usepackage[margin=0.9in]{geometry}
\renewcommand{\familydefault}{\sfdefault}
\usepackage{multicol}
\pagestyle{head}
\definecolor{c03}{HTML}{FFDDDD}
\header{AM 111 Class 11}{}{Skill Check}
\runningheadrule
\headrule
\usepackage{graphicx} % more modern
\usepackage{amsmath} 
\usepackage{amssymb} 
\usepackage{hyperref}
\usepackage{tcolorbox}

\begin{document}
 \pdfpageheight 11in 
  \pdfpagewidth 8.5in

\noindent Name: \rule{2.5in}{0.5pt}

\noindent Put away any notes or other materials, and work on this activity alone.

\noindent You'll receive feedback on your work and will complete a similar question on a future skill check.


\begin{questions}
\item 
Decide whether the equations form a cubic spline.

\[S(x) = \left\{\begin{array}{l l}
2x^3 + x^2 + 4x + 5 & \text{on }[0,1] \\
(x-1)^3 + 7(x-1)^2 + 12(x-1) + 12 & \text{on }[1,2] \\
\end{array}\right.\]

\vspace{6.5cm}

\item (Retake for Class 08)

For sufficiently large $k$, $\dfrac{\vert e_{k+1}\vert}{\vert e_k\vert^r} \approx C$

$C$ and $r$ are estimated from successive steps of a numerical algorithm below.

\begin{verbatim}
0.282347088755958 2.409420839653209
0.5 1.0
0.5 1.0
0.5 1.0
0.5 1.0
0.5 1.0
0.5 1.0
0.5 1.0
0.5 1.0
\end{verbatim}

Assuming a simple root, is this sequence generated by the bisection method, Newton's method, or the secant method?

\emph{These sequences were generated via the Python code that we used in class during Class07}

\end{questions}

\end{document}