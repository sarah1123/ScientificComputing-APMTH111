\documentclass[12pt,letterpaper,noanswers]{exam}
\usepackage[usenames,dvipsnames,svgnames,table]{xcolor}
\usepackage[margin=0.9in]{geometry}
\renewcommand{\familydefault}{\sfdefault}
\usepackage{multicol}
\pagestyle{head}
\header{AM 111 Class 02}{}{Floating point and binary, p.\thepage}
\runningheadrule
\headrule
\usepackage{siunitx}
\usepackage{graphicx} % more modern
\usepackage{amsmath} 
\usepackage{amssymb} 
\usepackage{hyperref}
\usepackage{tcolorbox}
\newcommand{\vc}[1]{\underline{#1}}
\begin{document}
 \pdfpageheight 11in 
  \pdfpagewidth 8.5in

\noindent 



\section{Preliminaries}
\begin{itemize}
\itemsep0pt
\item Problem set 01 is due on Friday at 5pm (submit via Gradescope and upload any source code to Canvas).
\item There will be a skill check in class during Class 03.  The problem info is below.
\item OH Thursdays 1-3pm and 4:30-6:30pm in Pierce G11 (AM Undergraduate Lounge).
\item Sarah has OH today from 2-3pm in Pierce G11.
\end{itemize}

\hrule
\vspace{0.2cm}
\noindent\textbf{Extra}

Accessing Python via FAS OnDemand

\vspace{0.2cm}
\hrule
\vspace{0.2cm}



\noindent\textbf{Big picture}

Today: how are numbers represented in computers?  How does this lead to approximation as an intrinsic component of doing math with a computer?

\vspace{0.2cm}
\hrule
\vspace{0.2cm}

\noindent \textbf{Skill check practice}

Convert $12.35$ to binary.


\vspace{0.2cm}
\hrule
\vspace{0.2cm}

\noindent \textbf{Skill check solution}

Convert the integer part and the decimal part separately.
\begin{multicols}{2}

$12/2 = 6 R0$

$6/2 = 3 R0$

$3/2 = 1 R1$

The integer part is $1$ so we keep that $1$ and stop the algorithm.

We have $(12)_{10} = (1100)_2 $

\columnbreak

$0.35 * 2 = 0.7 = 0.7 + 0$

$0.7*2 = 1.4 = 0.4 + 1$

$0.4*2 = 0.8 = 0.8 + 0$

$0.8*2 = 1.6 = 0.6 + 1$

$0.2*2 = 1.2 = 0.2 + 1$

$0.2*2 = 0.4 = 0.4 + 0$

$0.4*2 = 0.8 = 0.8 + 0$

So $(0.35)_{10} = (0.01\overline{0110})_2$

\end{multicols}

Combining, $(12.35)_{10} = (1100.01\overline{0110})_2$

\vspace{0.2cm}
\hrule
\vspace{0.2cm}

\noindent \textbf{Teams}

\begin{multicols}{3}
1) Kevin, Eli, Daniyal

2) Caitlin, Julia, Johan

3) Ibrahima, Nicole, Zachary

4) Sophie, Julia, Aidan

5) RJ, Brian, Nina

6) Kevin, Mack, Ray

7) Alex, Jack, Robert

8) Nini, Emma, Benjamin

9) Eletria, Tom, Basil

10) Serena, Jessica, Isaac

11) Shang, Esmé, Marissa

12) Eric, Cameron, Dani
\end{multicols}

%\section{Temporary}

%q1: %decimal to binary, 
%decimal to a binary floating point (with IEEE convention), machine epsilon

%q2: condition number

%q3: machine epsilon, %overflow, 
%info in the data type

%q4: calculus review (so not in class)

%q5: rounding vs chopping

\section{Floating point}

\begin{tcolorbox}
\begin{itemize}
    \item A floating point system can be characterized by four numbers:
\begin{itemize}
\itemsep0pt
    \item $\beta$ (base)
    \item $p$ (precision)
    \item $[L,U]$ (range for exponents)
\end{itemize}

\item A floating point number is of the form \[ \pm \left(d_0 + d_1\beta^{-1} + d_2 \beta^{-2} + ... + d_{p-1}\beta^{-(p-1)}\right) \beta^E\] where $0\leq d_i\leq \beta-1$ and $L\leq E \leq U$ are integers.

In base $10$ you would write $d_0.d_1d_2...d_{p-1} \times 10^E$.
\item $\pm$ is the sign, $d_0d_1d_2...d_{p-1}$ is the mantissa, $E$ is the exponent, and $p$ is the precision.
\item In a \textbf{normalized} system we require $d_0\neq 0$. \emph{i.e. There is a single nonzero digit to the left of the decimal point.}
\end{itemize}
\end{tcolorbox}

With your team, find space at a whiteboard. \textbf{Teams 1 and 2}, you'll take photos of your whiteboard work to share with the class.  Upload those in the Class 02 folder of the course \href{Google Drive}{}.  Find the link on Canvas.  

\begin{enumerate}
    \item Consider a normalized floating point system with $\beta = 10$, $p = 4$, $-5\leq E \leq 5$.  To practice with the system:
    \begin{parts}
    \item If the system were \textbf{not} normalized, what is the largest number you could write in this system?
    \item The system is normalized, though.  So find the largest number in the system.  
    
    This is called the \textbf{overflow level} (OFL).
    \item Write $+2.430\times 10^2$ as a decimal number.
    
    \emph{Notice that $4$ digits are included as part of the floating point representation.}
    \item Write $-0.0034$ as a floating point number in this system.  
    
    \emph{Note: this is a normalized system, so there is only one way to write this.}
    
    \emph{Include the sign, mantissa, and exponent when writing a floating point number}.
    
    \item Write $1$ as a floating point number in this system.  
    \item Write the smallest floating point number greater than $1$.
    \end{parts}
    \item Using the same system, we will examine a few more numbers.
    \begin{parts}
    \item Use your work above to find the smallest number $\epsilon$ such that $1+\epsilon > 1$.  This number is referred to as machine epsilon, $\epsilon_{\text{mach}}$
    \item How would you represent $\pi$ in this system?  \emph{If there are multiple options, add a label for your choice.}
    \item Find the smallest positive number represented in the system.  This is the underflow level (UFL).  How is it different from $\epsilon_{\text{mach}}$?
    

    
    \item Can you write $0$ in a normalized system?
    \end{parts}
    \item Now try some floating point arithmetic.
    \begin{parts}
    \item Compute $+1.000\times 10^{-5}+1.000\times 10^{-5}$
    \item Compute $1.0\times 10^{-5}+ 1$ within this system.
    \item Compute $(\num{1.0e-5} + 1) + (\num{1.0e-5}-1)$

\emph{This is called \textbf{catastrophic cancellation}.}
    \end{parts}
\end{enumerate}

\vspace{0.2cm}
\hrule
\vspace{0.2cm}


\noindent\textbf{Rounding}
\begin{tcolorbox}
\begin{itemize}
\itemsep0pt
    \item Let $\text{fl}(x)$ denote the floating point approximation to $x$.
    \item To convert from $x$ to $\text{fl}(x)$ one option is to \textbf{chop}, taking the first $p$ digits.  
    
%    Dropping the $p+1$st digit leads to a relative error of $\left\vert\dfrac{\text{fl}(x) - x}{x}\right\vert\leq \beta^{-(p-1)}$
    \item Another option is \textbf{rounding to nearest} by taking the floating point number nearest to $x$.  
    
 %  This leads to a relative error of $\left\vert\dfrac{\text{fl}(x) - x}{x}\right\vert\leq \frac{1}{2}\beta^{-(p-1)}$
    
    Note: If $x$ is equidistant between two floating point numbers (think of $1.5$ when rounding to the nearest integer), then the value of $d_{p-1}$ is used to determine whether to round up or down.  
    \item The maximum relative error between a number, $x$, and its floating point representation, $\text{fl}(x)}$, gives the accuracy of the floating point system.  One way to define machine epsilon is \[\epsilon_{\text{mach}} = \max\limits_{x\in[\text{UFL,OFL}]} \left\vert\dfrac{\text{fl}(x) - x}{x}\right\vert.\]

    \item An alternative definition is that $\epsilon_{mach}$ is the distance between $1$ and the next larger floating point number.
\end{itemize}
\end{tcolorbox}
\begin{enumerate}
\setcounter{enumi}{3}
    \item Practice with chop and round to nearest.  Let $\beta = 10$, $p =2$, $L = -3, U = 4$.

\begin{enumerate}
    \item 
\begin{tabular}{c|c|c}
     & chop & nearest \\
     \hline
  2.394   & & \\
2.251 & & \\
%2.389 & & \\
2.250 & & \\
\end{tabular}
\end{enumerate}
\item For $x = 2.394$, set up expressions for the relative error from chopping and from rounding.  

Which is worse in this case?

\end{enumerate}

\section{Binary}



\begin{tcolorbox}
\begin{itemize}
\itemsep0pt
    \item The floating point systems used in computers typically have $\beta = 2$ (binary), so the digits of the floating point numbers, $d_i = 0$ or $1$.  We will use $b_i$ interchangeably with $d_i$ when we are working in binary.
    \item A single binary digit is called a \textbf{bit}.
    \item In double precision, a floating point number has $64$ bits.  \[s e_1e_2...e_{11}b_1b_2...b_{52}\] where $s$ is the sign (1 bit), $e_1...e_{11}$ is the exponent (11 bits), and $b_1b_2...b_{52}$ are bits from the mantissa (52 bits).  Note that $b_1b_2...b_{52}$ are only the bits to the right of the decimal point (aka the binary point or radix point).
    \item A few exceptional numbers also need to be represented: NaN (not a number, i.e. 0/0), Inf (infinity, i.e. $1/0$), 0 (because $b_0 = 1$ is assumed, $0$ needs a special representation).  
\end{itemize}
\end{tcolorbox}




\noindent\textbf{Converting between binary and decimal}

\begin{enumerate}
\setcounter{enumi}{5}
    \item Converting from binary to decimal \begin{enumerate}
        \item $(11)_{10}$ (eleven in base ten) can be thought of as $1*10^1+1*10^0$.  If we are instead in base $2$, $(11)_2 = 1*2^1+1*2^0$. What does $(11)_2$ convert to in base 10?
        \item Write $(101)_2$ in base $10$.
        \item Write $(101.1)_2$ in base $10$.
    \end{enumerate}
    \item Converting integers from decimal to binary
    \begin{enumerate}
        \item $(16)_{10}$ = $(\_ \_ \_ \_ \_)_2$.  Find the base $2$ representation of $16$.
        \item Find the base $2$ representation for $17$.
    \end{enumerate}
\end{enumerate}

\begin{tcolorbox}
\textbf{Algorithm for base 10 integer to binary:}

In base $10$ when you divide an integer by $10$, the remainder is the right most digit of the integer: $17/10 = 1 R7$ where $7$ is the right most digit.

When you divide an integer by $2$ the remainder is the first digit of the binary representation:
$17/2 = 8 R1$ where $1$ is the right most bit

$(17)_{10} = (? ? ? ? ?)_2$.

$17/2 = 8 R1$.  $17/2 = (? ? ? ?.1)_2$. The integer part, $(? ? ? ?)_2$, is now $8$.  Repeat:



$8/2 = 4 R0$.  $17/2^2 = (? ? ?.01)_2$. The integer part, $(? ? ?)_2$, is now $4$.  Repeat:

$4/2 = 2 R0$.  $17/2^3 = (? ?.001)_2$. The integer part, $(? ?)_2$, is now $2$.  Repeat:

$2/2 = 1 R0$.  $17/2^4 = (1 .0001)_2$. When the integer part is $1$, we keep $1$ as the integer part and stop the algorithm.

Using the remainders, we find $17/2^4 = (1.0001)_2$ so $17 = (10001)_2$.

\end{tcolorbox}

\begin{enumerate}
\setcounter{enumi}{7}
    \item Use division by $2$ and keep the remainders to find $27$ in binary.
\end{enumerate}

\begin{tcolorbox}
\textbf{Algorithm for base 10 fractional or decimal part to binary:}

In base $10$ when you multiply a number $<1$ by $10$, the integer part is the left most digit of its decimal representation: $0.24*10 = 2.4$ where $2$ is the left most digit.

When you multiply a base $10$ fraction or decimal by $2$ the integer part is the first digit of the fractional part of the binary representation:

$\dfrac{1}{3} = (0. ? ? ? ? ?)_2$.

$\dfrac{1}{3}*2 = \dfrac{2}{3}$.  The integer part is $0$.  $\dfrac{1}{3}*2 = (0. ? ? ? ?)_2$. The fractional part is now $2/3$.  Repeat:

$2/3 * 2 = 4/3 = 1 + 1/3$.  $\dfrac{1}{3}*2^2 = (01. ? ? ?)_2$. The fractional part is now $\dfrac{1}{3}$.  Repeat:

Wait... we've already done $1/3$ so we know what happens from here.  This is a repeating binary expansion.  

$\dfrac{1}{3} = 0.\overline{01}$

\end{tcolorbox}

\begin{enumerate}
\setcounter{enumi}{7}
    \item Use multiplication by $2$ and keep the integer parts to find $0.3$ in binary.
    \item Express $27.3$ as a floating point number using double precision and rounding to nearest.
    \begin{parts}
    \item First, convert $27.3$ to binary.
    \item Example notation for expressing the floating point number (have 52 bits in the box):
\[\text{fl}(\frac{1}{3}) = +1.\ \framebox[11cm][l]{0101010101010101010101010101010101010101010101010101}\times 2^{-2}\]

Follow this notation (and use the round to nearest rule) to write $27.3$.
    \end{parts}
    
    
    \item 

Use this notation to write $27.3$ as a floating point number (with $\beta = 2$, $p = 53$)
\end{enumerate}

\eject

\noindent\textbf{Some answers}
\begin{enumerate}
    \item a) $+9999\times 10^5$.  b) $+9.999\times 10^5$.  c) $243$.  d) $-3.400\times 10^{-3}$.  e) $+1.000\times 10^0$.  f) $1.001\times 10^0$ 
    \item a) $1.001\times 10^0$ is the next number, so $\epsilon = 1.000\times10^{-3}$.  b) $\pi = 3.141592...$ so $3.142$ (round) or $3.141$ (truncate) c) $+1.000\times10^{-5}$ so $0.00001$. d) $0 = 0.0000$ but there has to be a nonzero number to the left, so not clear how to, no.
    \item a) $+2.000\times10^{-5}$ b) $1.000 + 0.00001 = 1.00001$ but we stop at $4$ digits so $+1.000\times 10^0$.  c) $-1.000 + 0.00001$ is $-1.000$ so $0$.
    \item \begin{tabular}{c|c|c}
     & chop & nearest \\
     \hline
  2.394   & 2.3& 2.4\\
2.251 & 2.2& 2.3 \\
2.250 & 2.2 & 2.2 \\
\end{tabular}
$2.25$ was between $2.2$ and $2.3$ so look to the $3$ to decide which way to round.
\item chop: $\vert(2.394-2.3)/2.394\vert = 0.94/2.394$.  round: $\vert(2.394-2.4)/2.394\vert = 0.006/2.394$.  chopping is worse (something like 40\% !?)
\item a) $(11)_2 = 1+2 = 3$. b) 4+0+1 = 5.  c) $5+ 1*2^{-1} 5.5$.
\item a) $16 = 2^4$ so $(10000)_2$. b) $17 = 16+1$ so $(10001)_2$.
\item $27/2 = 13 R1$.  $13/2 = 6 R1$.  $6/2 = 3 R0$.  $3/2 = 1 R1$.  $(11011)_2$.  Check: $16+8+0+2+1 = 27.$
\item $0.3*2 = 0.6$ so $(0.0...)_2$.  $0.6*2 = 1.2$ so $(0.01...)_2$.  $0.2*2 = 0.4$ so $(0.010...)_2$.  $0.4*2 = 0.8$ so $(0.0100...)_2$.  $0.8*2 =1.6$ so $(0.01001...)_2$.  We have seen $0.6$ onward before so $1001$ will repeat.  $(0.0\overline{1001})_2$.  Check: $x = (0.0\overline{1001})_2$.  $2x = (0.\overline{1001})_2$.  $2*2^4x = (1001.\overline{1001})_2$.  $(1001)_2 = 9$ so $2x + 9 = 32x$.  $30x = 9 \Rightarrow x = 3/10 = 0.3$
\item a) $27.3 = 27+0.3 = (11011.0\overline{1001})_2$  b) $+1.1011010011001100\times 2^4$ <-- 12 bits after the radix point. $12+8*5 = 52$ so make 5 copies of the last 8:

$+1.10110100110011001100110011001100110011001100110011001100\times 2^4$.  For rounding, the next bits are $11$ so round up

$+1.\framebox[11.7cm][l]{10110100110011001100110011001100110011001100110011001101}\times 2^4$
\end{enumerate}
\end{document}