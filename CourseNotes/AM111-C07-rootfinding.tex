\documentclass[12pt,letterpaper,noanswers]{exam}
\usepackage[usenames,dvipsnames,svgnames,table]{xcolor}
\usepackage[margin=0.9in]{geometry}
\renewcommand{\familydefault}{\sfdefault}
\usepackage{multicol}
\pagestyle{head}
\header{AM 111 Class 07}{}{Solving equations, p.\thepage}
\runningheadrule
\headrule
\usepackage{siunitx}
\usepackage{graphicx} % more modern
\usepackage{amsmath} 
\usepackage{amssymb} 
\usepackage{hyperref}
\usepackage{tcolorbox}
\usepackage{enumitem}
\def\mbf{\mathbf}
\newcommand{\vc}[1]{\boldsymbol{#1}}
\def\dsst{\displaystyle}
\DeclareMathOperator*{\argmin}{arg\,min} % thin space, limits underneath in displays


\begin{document}
 \pdfpageheight 11in 
  \pdfpagewidth 8.5in

\noindent 

\section*{Preliminaries}

\begin{itemize}
\itemsep0pt
\item Problem set 03 is due on Friday at 5pm (submit via Gradescope: include pdfs of all code/output on Gradescope.  Upload any source code to Canvas).
\item Problem set 03 includes some ``time permitting'' problems.  If your total time spent on the course outside of class reaches 10 hours in the week then you are encouraged to skip problems.  If you are not in that situation, you are expected to complete the problems.
\item There will be a skill check in class during Class 08.  The problem info is below.
\item Find all OH on Canvas.
\end{itemize}



\noindent\textbf{Big picture}

Today: Algorithms for approximating solutions to nonlinear equations.

\vspace{0.2cm}
\hrule
\vspace{0.2cm}

\noindent \textbf{Skill check practice}
\begin{questions}
\item For sufficiently large $k$, $\dfrac{\vert e_{k+1}\vert}{\vert e_k\vert^r} \approx C$


$C$ and $r$ are estimated from successive steps of a numerical algorithm below.

\begin{verbatim}
0.11369060817278044 3.329460092761372
0.1573815133502217 0.8941370047142863
1.3248374877542113 2.125788830275385
0.11466352084798608 1.4051502969581708
1.010392350228528 1.7188363535167348
0.18863719882541435 1.5779653647657021
\end{verbatim}

Assuming a simple root, is this sequence generated by the bisection method, Newton's method, or the secant method?

\item The skill from the Class 04 handout (Skill Check C05).
\end{questions}


\vspace{0.2cm}
\hrule
\vspace{0.2cm}

\noindent \textbf{Skill check solution}
\begin{questions}
\item In the later steps, $r\approx 1.5$, i.e. it is superlinear but not approaching quadratic.  This is the secant method.

\item See the past handout.
\end{questions}
\vspace{0.2cm}
\hrule
\vspace{0.2cm}

\noindent \textbf{Teams}
\begin{multicols}{3}
1. Eletria, Benjamin, Marissa

2. Cameron, Basil, Emma

3. RH, Eric, Esmé

4. Nini, Ray, Dani

5. Jack, Mina, Ivonne

6. Alex, KevinG, Shang

7. Jessica, Johan, Tom

8. Nina, Robert, Padraig

9. KevinC, Alex, Eli

10.  Aidan, Daniyal, Zachary

11. Ghedion, JuliaK, JuliaM

12. Mack, Brian

13. Caitlin, Sophie

\end{multicols}

\subsection*{Newton's method}
\begin{tcolorbox}
Assume $f$ is a differentiable function with a zero at $x^*$.  Let $x_k$ be an approximation to $x^*$ such that $f'(x_k)\neq 0$ and $\vert x^*-x_k\vert$ is small.
\begin{itemize}
\itemsep0pt
    \item $ 0 = f(x^*) \approx f(x_k) + (x^*-x_k)f'(x_k)$.  Rearranging, $x^* \approx x_k - \dfrac{f(x_k)}{f'(x_k)}$.
    \item Set $x_{k+1} = x_k - \dfrac{f(x_k)}{f'(x_k)}$.  This is \textbf{Newton's method}.
    \item This method converges quadratically: $\lim\limits_{k\rightarrow\infty} \dfrac{\vert e_{k+1}\vert }{\vert e_k\vert^2 } = \dfrac{\vert f''(x^*)\vert}{2\vert f'(x^*)\vert} = C < \infty$
\end{itemize}
\end{tcolorbox}
\begin{enumerate}[resume=classQ]
    \item (Sauer \S1.4 1a) Apply two steps of Newton's method with initial guess $x_0 = 0$ for $f(x) = x^3+x-2$
    \vspace{1in}
    
  
\end{enumerate}

\subsection*{Secant method}
\begin{tcolorbox}
Assume $f$ is a continuous function.  Use $\dfrac{f(x_k)-f(x_{k-1})}{x_k-x_{k-1}}$ in place of $f'(x_{k})$.  

Newton's method becomes \[x_{k+1} = x_k - f(x_k)\dfrac{x_k-x_{k-1}}{f(x_k) - f(x_{k-1})}\]
\end{tcolorbox}

\subsection*{Summary}
\begin{tabular}{p{0.3\linewidth} p{0.3\linewidth} p{0.3\linewidth}}
bisection & Newton's & secant \\
\hline
always converges & may not converge & may not converge\\
sign of $f$  & needs $f$ and $f'$ & just needs $f$ \\
$f$ continuous & $f$ differentiable & $f$ continuous\\
linear convergence & quadratic convergence & superlinear ($<$ quadratic) \\
needs initial interval & one initial value& two initial values\\
no higher-dim'l version & higher dim'l version & higher dim'l version \\
\end{tabular}

\begin{enumerate}[resume=classQ]
\item (Health \S5 5.24)

List one advantage and one disadvantage of the secant method compared with Newton's method for solving a nonlinear equation in one dimension.
\vspace{0.5in}

\item (Heath \S5 5.32) 

Think about how you might combine the idea of  a bracket around the solution from the bisection method with the secant method so that the method will still converge even if it starts far from a root.

\emph{This is called a safeguarded method}
\vspace{1in}

\end{enumerate}

\subsection*{Order of convergence}

\begin{tcolorbox}
Let $x_0, x_1, ..., x_k, ...$ be the sequence produced by a root-finding algorithm.  Ideally it will converge as fast as possible.

We have $e_k = x_k - x^*$, the error at the $k$th step.  If the sequence of $x_k$ converges to $x^*$ as $k\rightarrow\infty$, there exists $r\geq 1$, $C>0$ such that $\lim\limits_{k\rightarrow\infty}\dfrac{\vert e_{k+1}\vert}{\vert e_k\vert^r} = C$ where $r$ is the \textbf{order of convergence} and $C$ is the rate of convergence.
\end{tcolorbox}
\begin{enumerate}[resume=classQ]
\item For sufficiently large $k$, $\dfrac{\vert e_{k+1}\vert}{\vert e_k\vert^r} \approx C$.
\begin{parts}
\item $\vert e_{k+1}\vert\approx C\vert e_k\vert^r$.  Write a similar expression for $\vert  e_{k+2}\vert $.
\vspace{0.5in}

\item Use the ratio of your expressions to show that $\log \left\vert\dfrac{e_{k+2}}{e_{k+1}}\right\vert \approx r \log \left\vert\dfrac{e_{k+1}}{e_{k}}\right\vert$
\vspace{1in}

\item Rearrange to produce an expression that approximates $r$.
\end{parts}
\end{enumerate}

\eject
\section*{Numerical Case Study: Root finding methods}

We are going to use bisection, Newton's method, and the secant method.  Find the code on Canvas.  Investigate four functions:
\begin{enumerate}
\item[f1] $f_1(x) = x^2 - x - 6$ (roots -2,3)
\item[f2] $f_2(x) = x^3 - x^2$ (roots 0,1)
\item[f3] $\dsst{f_3(x) = x^3 - 2x^2 + \frac{4x}{3} - \frac{8}{27}}$ (root 2/3)
\item[f4] $f_4(x) = x - 2^{-x}$
\end{enumerate}
\begin{enumerate}[resume=classQ]
    \item Download the files from Canvas.
    \item Check that the functions and derivatives defined in the methods Jupyter Notebook correspond to the functions above.
\end{enumerate}



\subsection*{Bisection Method}

\noindent
Try the bisection method. The code includes parameters that you might want to change while going through the tasks below, and generates a figure of the estimate of the order $r$ of the method.  Note that $[a,b]$ is your initial interval for estimating the root in this code. 

\begin{enumerate}[resume=classQ]
\item Find starting intervals that lead to finding the roots of each function.

\vfill

\item Is there an issue with a root of $f2$? Why?

\vfill

%\item Save a screenshot of your answers on this page and a plot of your estimate of $\alpha$ for $f1$.

%\vfill

\end{enumerate}










\subsection*{Newton's Method}

\noindent
Now try Newton's method.  Note that $x$ is your initial estimate for the root in this code.

\begin{enumerate}[resume=classQ]
\item Find initial guesses that converge to each root for each function, as well as initial guesses that don't converge for each function.

\vfill

\item For $f1$, do you see a doubling of zeros in the accuracy? Look at the printout of the estimate of the root.

\vfill

\pagebreak
\item For the $f2$ double root at zero, the estimate for $r$ should be around 1.  Newton's method only converged quadratically for simple roots.  To increase the convergence at a double root, observe that $f^{1/2}$ has a simple root.  Find $\dfrac{\sqrt{f}}{\frac{d}{dx}\sqrt{f}}$.  This leads to a modified Newton's method: $x_{k+1} = x_k - 2\dfrac{f(x_k)}{f'(x_k)}$.  

\emph{Another option for a double root is to observe that $g(x) = \dfrac{f(x)}{f'(x)}$ will have a single root at the same location.}

\vfill

\item Estimate $r$ for $f2$ using modified Newton's method.


\end{enumerate}








%\pagebreak
\section*{Secant Method}

\noindent
Next work with the secant method.  Note that $xOld$ and $x$ are your two initial estimates for the root in this code.

\begin{enumerate}[resume=classQ]
\item Find initial guesses that converge to each root for each function.

\vfill

\item Find initial guesses that don't converge to a root for each function.

\vfill

\item Make a plot of your estimate of $r$ for $f4$.

\vfill

\end{enumerate}

\end{document}