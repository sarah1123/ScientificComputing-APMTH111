\documentclass[12pt,letterpaper,noanswers]{exam}
\usepackage[usenames,dvipsnames,svgnames,table]{xcolor}
\usepackage[margin=0.9in]{geometry}
\renewcommand{\familydefault}{\sfdefault}
\usepackage{multicol}
\pagestyle{head}
\header{AM 111 Class 06}{}{Solving equations, p.\thepage}
\runningheadrule
\headrule
\usepackage{siunitx}
\usepackage{graphicx} % more modern
\usepackage{amsmath} 
\usepackage{amssymb} 
\usepackage{hyperref}
\usepackage{tcolorbox}
\usepackage{enumitem}
\def\mbf{\mathbf}
\newcommand{\vc}[1]{\boldsymbol{#1}}
\DeclareMathOperator*{\argmin}{arg\,min} % thin space, limits underneath in displays


\begin{document}
 \pdfpageheight 11in 
  \pdfpagewidth 8.5in

\noindent 

\section*{Preliminaries}

\begin{itemize}
\itemsep0pt
\item Problem set 03 is due on Friday at 5pm (submit via Gradescope: include pdfs of all code/output on Gradescope.  Upload any source code to Canvas).
\item Problem set 03 includes some ``time permitting'' problems.  If your total time spent on the course outside of class reaches 10 hours in the week then you are encouraged to skip problems.  If you are not in that situation, you are expected to complete the problems.
\item There will be a skill check in class during Class 07.  The problem info is below.
\item Find all OH on Canvas.
\end{itemize}



\noindent\textbf{Big picture}

Today: Algorithms for approximating solutions to nonlinear equations.

\vspace{0.2cm}
\hrule
\vspace{0.2cm}

\noindent \textbf{Skill check practice}
\begin{questions}
\item Let $f(x) = x^2 -x -1$ and $x_0 = 0$.  Apply one step of Newton's method.
\item The skill from the Class 03 handout (Skill Check C04).
\end{questions}


\vspace{0.2cm}
\hrule
\vspace{0.2cm}

\noindent \textbf{Skill check solution}
\begin{questions}
\item $x_1 = x_0 - f(x_0)/f'(x_0)$.  

$f'(x) = 2x - 1$.

$f(0) = -1$, $f'(0) = -1$.  $x_1 = 0 - -1/(-1) = -1$

$f(-1) = -1$, $f'(-1) = -3$.  $x_2 = -1 - -1/(-3) = -4/3$

\item See the past handout.
\end{questions}
\vspace{0.2cm}
\hrule
\vspace{0.2cm}

\noindent \textbf{Teams}

\begin{multicols}{3}
1) Kevin, Eli, Daniyal, Jessica

2) Caitlin, Julia, Johan

3) Mai, Zachary, Padraig, Ghedion

4) Sophie, Julia, Aidan

5) RJ, Brian, Nina

6) Kevin, Mack, Ray

7) Alex, Jack, Robert

8) Nini, Emma, Benjamin

9) Eletria, Tom, Basil

11) Shang, Esmé, Marissa

12) Eric, Cameron, Dani

13) Alex, Ivonne, Mina
\end{multicols}

\textbf{Teams 12 and 13}, post photos of your work to the class Google Drive.  Find the link on Canvas.

\noindent See Sauer chapter 1, Heath chapter 5, Greenbaum and Chartier chapter 4, Koumoutsakos notes from ETH lecture 2.



\section*{Root finding}
\subsection*{Introduction}


\begin{tcolorbox}
(from Koutmoutsakos et al Lecture 2)

\begin{itemize}
\itemsep0pt
    \item A nonlinear equation $g(x) = h(x)$ can be rewritten as $f(x) = 0$.
    \item Solving $f(x) = 0$ is the problem of finding a zero or \textbf{root} of the function $f(x)$, i.e. a value $x^*$ such that $f(x^*) = 0$, so the problem is referred to as a \textbf{root finding} problem.
\end{itemize}
\end{tcolorbox}

\begin{tcolorbox}
\begin{itemize}
\itemsep0pt
    \item Root finding methods typically use \textbf{iteration}, starting from an initial guess $x_0$, and produce a sequence, $x_0, x_1, ..., x_k$ that converges to $x^*$ as $k\rightarrow \infty$.
    \item The algorithm is stopped according to a stopping criteria, such as $\vert x_k - x_{k+1}\vert < h$ where $h$ is a pre-selected tolerance.
\end{itemize}
\end{tcolorbox}

\subsection*{The bisection method}
%Given a function $f(x)$ and an initial interval $[a,b]$ with $f(a)f(b)<0$:

  %\fbox{
    \begin{minipage}{15cm}
 %     \subsection*{The Bisection Method}
      Given a continuous function $f(x)$ and an initial interval $[a,b]$ so that $f(a)f(b) < 0$.
      \begin{align*}
        \mbf{while} & \quad (b-a)/2 > TOL \ \mbf{ and }\ k < k_{\text{max}}\hspace{4in}\\
      & c_k=(a+b)/2 \\
      & \mbf{if} \quad f(c_k) = 0 \quad \mbf{stop}, \mbf{ end}\\
      & \mbf{if} \quad f(a)f(c_k)<0\\
      & \qquad b=c_k \\
      & \mbf{else} \\ 
      & \qquad a=c_k \\
      & \mbf{end}\\ 
      & k = k + 1 \\
      \mbf{end} &
      \end{align*}
      The interval $[a,b]$ is updated on each pass through the loop, and contains a root.
      
      Use $x_k = (a+b)/2$ to approximate the location of the root 
   \end{minipage}
%}

\begin{enumerate}[resume=classQ]
\item Find the roots of $x^2-x-6$ exactly (i.e. using tools you already know for solving quadratic equations / factoring etc)
\vspace{1in}

\item Use the bisection method to produce successive approximations, $x_0, x_1, ...$ to a root of $x^2-x-6$ in $[0,8]$.
\vspace{1in}

\end{enumerate}


\begin{tcolorbox}
Suppose that $f\in C[a,b]$ (i.e. $f$ is continuous on the interval $[a,b]$), and $f(a)f(b)<0$.
\begin{itemize}
\itemsep0pt
    \item (Bolzano's theorem) $f(x)$ has a root in the interval $[a,b]$.  This is a corollary of the intermediate value theorem.
    \end{itemize}
\end{tcolorbox}
\begin{tcolorbox}
\begin{itemize}
    \item (intermediate value theorem) for all $y$ in between $f(a)$ and $f(b)$ there is an $x^*\in(a,b)$ such that $f(x^*) = y$.

    \item The bisection method generates a sequence $\left\{x_k\right\}_{k=1}^\infty$ approximating a zero, $x^*$, of $f$ with $\vert x_k - x^*\vert < \dfrac{b-a}{2^{k+1}}$ when $k\geq 1$.
\end{itemize}
\end{tcolorbox}
\begin{enumerate}[resume=classQ]
    \item How many steps would be needed to find the root of $x^2-x-6$ in the interval $[0,10]$ to guarantee accuracy to $3$ decimal places?  
    


\emph{Consider a solution correct to within $q$ decimal places if the error $\vert x_k - x^*\vert < 0.5*10^{-q}$}
\vspace{1in}


    \item Let $e_k = x_k - x^*$.  Assuming $\vert e_k \vert \approx\dfrac{b-a}{2^{k+1}}$, find an approximation for $\dfrac{ \vert e_{k+1} \vert}{ \vert e_k \vert}.$
        \vspace{1in}
        
    \item When $\lim\limits_{k\rightarrow\infty} \dfrac{ \vert e_{k+1} \vert}{ \vert e_k \vert} = C$ with $0<C<1$, the convergence is linear with rate $C$.  
    
    \emph{The error is going down by a constant factor at each step.}
    
    When $C = 0$ it is superlinear (faster than linear) and when $C = 1$ it is sublinear (slower than linear).  
What is the order of convergence, and the convergence rate, for the bisection method?

\vspace{0.5in}

\end{enumerate}


\subsection*{Important considerations}
\begin{tcolorbox}
\begin{itemize}
\itemsep0pt
    \item When does $f(x) = 0$ have solutions?
    
    In 1D, see Bolzano's theorem; in the multidimensional case there is not an easy answer
    \item Is our method guaranteed to converge to a solution?
    \item How fast is the method converging?
    
    A method has an \textbf{order of convergence} and an \textbf{asymptotic error constant}.
    
    %  The bisection method improves by a factor of $1/2$ at each step.  
\end{itemize}
\end{tcolorbox}



\begin{enumerate}[resume=classQ]
\itemsep0pt
    \item Is the bisection method guaranteed to converge?  
\end{enumerate}

\subsection*{Iterated methods}

\begin{tcolorbox}
\begin{itemize}
\itemsep0pt
    \item Consider a sequence $x_k = g(x_{k-1})$.  $g$ is a \textbf{map} from $x_{k-1}$ to $x_k$.  
    \item A point $x^*$ such that $g(x^*) = x^*$ is called a \textbf{fixed point} of the map.
    \item Some fixed points are \textbf{attracting} or \textbf{stable}.  If you start nearby, at $x_0 = x^* + \Delta x_0$, the sequence $x_0, x_1, ...$ will converge to the fixed point.
\end{itemize}
\end{tcolorbox}

\begin{enumerate}[resume=classQ]
\itemsep0pt
    \item Let $g(x) = x - \frac{1}{10}(x^2-x-6)$.
    \begin{parts}
    \item Show that $g(x) = x$ at the root of $f(x) = x^2-x-6$.
    \vspace{1in}
    
    \item Choose a starting point of $x_0 = 0$ or $x_0 = 8$.  Use a calculator, and iterate the map to produce the sequence $x_0, x_1, ...$
    \vspace{1in}
    
    \item Use a Taylor expansion to first order, and simplify, to provide an approximation for $\dfrac{\vert \Delta x_k\vert}{\vert \Delta x_{k-1}\vert}$, where $x_k = x^* + \Delta x_k$.  Use the $g(x)$ above and let $x^* =3$.
    
    
    \[x_k = x^* + \Delta x_k= g(x_{k-1}) \approx \text{Taylor expansion of } g(x^*+\Delta x_{k-1}) \text{ about }x^*\]
\vspace{1in}

\part Let $g(x) = x - a(x^2-x-6)$.  Assuming $X^* = 3$, for what range of $a$ is $\dfrac{\vert \Delta x_k\vert}{\vert \Delta x_{k-1}\vert} < 1$?  If possible, find a value of $a$ so that it is $0$.
\vspace{0.5in}

    \end{parts}
\end{enumerate}
\subsection*{Newton's method}
\begin{tcolorbox}
Assume $f$ is a differentiable function with a zero at $x^*$.  Let $x_k$ be an approximation to $x^*$ such that $f'(x_k)\neq 0$ and $\vert x^*-x_k\vert$ is small.
\begin{itemize}
\itemsep0pt
    \item $ 0 = f(x^*) \approx f(x_k) + (x^*-x_k)f'(x_k)$.  Rearranging, $x^* \approx x_k - \dfrac{f(x_k)}{f'(x_k)}$.
    \item Set $x_{k+1} = x_k - \dfrac{f(x_k)}{f'(x_k)}$.  This is \textbf{Newton's method}.
    \item This method converges quadratically: $\lim\limits_{k\rightarrow\infty} \dfrac{\vert e_{k+1}\vert }{\vert e_k\vert } = \dfrac{\vert f''(x^*)\vert}{2\vert f'(x^*)\vert} = C < \infty$
\end{itemize}
\end{tcolorbox}
\begin{enumerate}[resume=classQ]
    \item (Sauer \S1.4 1a) Apply two steps of Newton's method with initial guess $x_0 = 0$ for $f(x) = x^3+x-2$
    \vspace{1in}
    
    \item (Heath \S 5.4) Match the error sequence with convergence information.
    
    \begin{enumerate}
    \item quadratic convergence
    \item linear convergence with $C = 10^{-1}$
    \item linear convergence with $C = 10^{-2}$
    \item superlinear ($<$ quadratic)
    \end{enumerate}
    
    \begin{enumerate}
    \item[(i)] $10^{-2},10^{-4},10^{-6},10^{-8},...$
    \item[(ii)] $10^{-2},10^{-4},10^{-8},10^{-16},...$
    \item[(iii)] $10^{-2},10^{-4},10^{-7},10^{-12},...$
    \item[(iv)] $10^{-2},10^{-3},10^{-4},10^{-5},...$
    \end{enumerate}
\end{enumerate}

\subsection*{Secant method}
\begin{tcolorbox}
Assume $f$ is a continuous function.  Use $\dfrac{f(x_k)-f(x_{k-1})}{x_k-x_{k-1}}$ in place of $f'(x_{k})$.  

Newton's method becomes \[x_{k+1} = x_k - f(x_k)\dfrac{x_k-x_{k-1}}{f(x_k) - f(x_{k-1})}\]
\end{tcolorbox}

\subsection*{Summary}
\begin{tabular}{p{0.3\linewidth} p{0.3\linewidth} p{0.3\linewidth}}
bisection & Newton's & secant \\
\hline
always converges & may not converge & may not converge\\
sign of $f$  & needs $f$ and $f'$ & just needs $f$ \\
$f$ continuous & $f$ differentiable & $f$ continuous\\
linear convergence & quadratic convergence & superlinear ($<$ quadratic) \\
needs initial interval & one initial value& two initial values\\
no higher-dim'l version & higher dim'l version & \\
\end{tabular}

\begin{enumerate}[resume=classQ]
\item (Health \S5 5.24)

List one advantage and one disadvantage of the secant method compared with Newton's method for solving a nonlinear equation in one dimension.
\vspace{0.5in}

\item (Heath \S5 5.32) 

Think about how you might combine the idea of  a bracket around the solution from the bisection method with the secant method so that the method will still converge even if it starts far from a root.

\emph{This is called a safeguarded method}

\vspace{0.5in}
\end{enumerate}
\end{document}