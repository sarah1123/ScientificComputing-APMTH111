\documentclass[12pt,letterpaper,noanswers]{exam}
\usepackage[margin=0.9in]{geometry}
\renewcommand{\familydefault}{\sfdefault}
\usepackage{multicol}
\pagestyle{head}
\header{AM 111}{}{Project Log}
\runningheadrule
\headrule
\usepackage{graphicx} % more modern
\usepackage{amsmath} 
\usepackage{amssymb} 
\usepackage{enumitem}
\usepackage{hyperref}
\usepackage{tcolorbox}

\begin{document}
 \pdfpageheight 11in 
  \pdfpagewidth 8.5in

\noindent Components of the weekly project log (every week):
\begin{enumerate}[resume=class]
\item For the project log, answer the following questions:
\begin{parts}
\item What was the main purpose of your project work this week?
\item What resources or references did you consult this week?
\item What did you accomplish on the project this week?
\end{parts}
\item With the project log, submit the following info:
\begin{itemize}
    \item log when you worked (date and times)
\item with whom you were working (if anyone)
\item what you were doing on the project during that time period
\end{itemize}
\item Upload all of your project work from the week: any code you've worked on, and any notes you've made. If the work is joint and you were involved, go ahead an upload it.  Do \textbf{not} re-submit notes from a prior week.
\end{enumerate}

\noindent This week: 
\begin{enumerate}[resume=class]

\item As one part of your project work this week, read a paper (or other resource set) that is central to your project.  Take your own notes on your reading. 

Identify the paper or resource.

\begin{parts}
\item Write an individual summary of the main ideas and purpose of the reading, expressing your own understanding of what you have read.  Include a summary of how the authors say it connects to related work.
\item Look at the figures in the paper.  Note which ones seem like they would be possible for you to replicate (or to create analogs of if you'll be using different data).
\item Identify what aspects of the analysis you currently understand and can replicate.  To the extent you're currently able to, provide your own version of the derivations of formulas, filling in steps that were missing in the resource.
\item Create a comprehensive list of vocabulary, mathematical details, and other elements that you do not yet fully understand.
\end{parts}

\end{enumerate}

\noindent Project submissions:
\begin{itemize}
    % \item Friday Nov 4: Project proposal (group submission).  
    
    % Timeline of project work and deliverables (individual submission).
    \item Friday Nov 11: Weekly project update and work log (individual submission)
    
    Close read of project paper with notes (individual submission)

    \item Friday Nov 18: Weekly project update and work log (individual submission).  
    
    Draft presentation slides.  Minimum of 1 slide per team member, maximum of 2 slides per team member.  (individual or group submission)
    \item Due Friday Nov 25: Weekly project update and work log (individual submission).  
    \item Tuesday Nov 29 or Thurs Dec 1: Presentation explaining your project topic to the class (group submission).  
    
    1 person team: 2 minutes.  2 person team: 3.5 minutes.  3 person team: 4.5 minutes.
    \item Due Friday Dec 2: Weekly project update and work log (individual submission).  
    
    Draft final project summary (group submission).
    \item Due Wednesday Dec 7: Peer review on final project summary (individual submission)
    \item Due Friday Dec 9: Weekly project update and work log (individual submission).
    \item Due Thursday Dec 15: Final project summary (group submission).  
    
    Individual project report (individual submission).
\end{itemize}


\end{document}