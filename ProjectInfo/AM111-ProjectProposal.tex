\documentclass[12pt,letterpaper,noanswers]{exam}
\usepackage[margin=0.9in]{geometry}
\renewcommand{\familydefault}{\sfdefault}
\usepackage{multicol}
\pagestyle{head}
\header{AM 111}{}{Project Proposal / Work Statement}
\runningheadrule
\headrule
\usepackage{graphicx} % more modern
\usepackage{amsmath} 
\usepackage{amssymb} 
\usepackage{hyperref}
\usepackage{tcolorbox}

\begin{document}
 \pdfpageheight 11in 
  \pdfpagewidth 8.5in

%The goal of the final project is for you to apply numerical analysis or scientific computing techniques to an applied or theoretical problem, or to conduct an investigation into a method not discussed in class.

Due Friday Nov 4th at 5pm: project proposal / work statement

\begin{enumerate}
    \item Aim: What will be the purpose of the project?  Create a defined question.  Go beyond generalities to include one or more specific goals.
    \item Components: Identify the tasks that your team will need to accomplish in order to satisfy your project goals.
    \item Knowledge building: Provide detailed information about how the team will engage in knowledge building.  What tools and methods do you need to understand?  What resources will you use?
    \item Computational methods: What computational method(s) does your team plan to implement yourselves?  What method(s) will you incorporate by using built-in options in Python?
    \item Division of responsibility: How will your team apportion work to team members?  What will be the responsibilities of each team member?
%    \item Weekly progress: Provide expectations of weekly progress.
    \item Communication: How does your group plan to communicate with each other and exchange work or ideas?  When/where/how often will your group meet?
    \item Scope: If this turns out to be more (or less) work than you expected, how can you adjust the project to take on more (or less)?

\end{enumerate}

% \noindent\textbf{Timeline}
% \begin{enumerate}
%     \item [Oct 29 - Nov 4:] Problem Set 08 + Proposal
%     \item [Nov 5 - Nov 11:] Project work + close read of a reference
%     \item [Nov 12 - Nov 18:] Problem Set 09 + Project work + draft slides
%     \item [Nov 19 - Nov 25:] Project + Thanksgiving
%     \item [Nov 26 - Dec 2:] Quiz 02 + Presentation + Draft project summary
%     \item [Dec 3 - ]
%     \item Project work (reading week and final exam period: due Dec 15)
% \end{enumerate}


% \begin{enumerate}
% \itemsep0pt
%     \item The (short) preliminary project proposals are due on Thursday at 5pm.
%     \item Each individual group member is responsible for submitting a different one, so your group will propose as many projects as the group has members.
%     \item Because these papers are popular, your group can propose either the flight reconstruction paper or the deep learning paper, but not both (as one of your three proposals).
%     \item Each team will work on a different project (with only one team per paper or resource).
% \end{enumerate}


% \noindent For each preliminary project, answer the following questions:
% \begin{itemize}
% \itemsep0pt
% \item What will be the aim or purpose of the project?
% \item What paper(s) or resource(s) would you like to base your project work around?
% \item How does the intended project incorporate mathematical topics related to this course?
% \item What method or methods do you intend to implement as part of the project?
% \item What do you think you will you need to learn to be able to complete your project?
% \end{itemize}

% \noindent For the project topic that your group will be working on:


% Due Friday Nov 11:
% \begin{parts}
% \part As one part of your project work this week, you will read a paper or other resource that is closely related to your project.  Take your own notes on your reading.

% \begin{itemize}
%     \item Write an individual summary of the main ideas and purpose of the reading, expressing your own understanding of what you have read.
%     \item Create a comprehensive list of the details, vocabulary, and other elements that you do not yet fully understand.
%     \item If relevant, identify what aspects of the analysis you could currently replicate on your own.
%     \item 
    
%     Read the abstract of your project paper carefully.  Note any content that you don't yet fully understand, terms that don't make sense, and sentences that are confusing.
%     \item Read the introduction of the paper. Note of any background knowledge that you don't yet have.
%     \item Based on the abstract and introduction, write a brief summary of the purpose of the paper and how the authors say it connects to related work.
%     \item Look at the equations in the paper (if any).  Note the equations that you already have the knowledge / tools to understand / replicate.
%     \item Look at the figures in the paper.  Note which ones seem like they would be easier to replicate (or to create analogs of if you'll be using different data).
% \end{itemize}

% \part 
% Read part of the paper closely and take notes individually (do not submit shared notes).  Upload a pdf of your notes to Canvas as part of your project log.

% Begin your work on the project: look up terms, make a list of questions you'd like to ask the course staff, work on replicating an initial analysis, find data to work with initially, etc.

% \end{parts}

\noindent Project submissions:
\begin{itemize}
    \item Friday Nov 4: Project proposal (group submission).  
    
    Timeline of project work and deliverables (individual submission).
    \item Friday Nov 11: Weekly project update and work log (individual submission)
    
    Close read of project paper with notes (individual submission)

    \item Friday Nov 18: Weekly project update and work log (individual submission).  
    
    Draft presentation slides.  Minimum of 1 slide per team member, maximum of 2 slides per team member.  (individual or group submission)
    \item Due Friday Nov 25: Weekly project update and work log (individual submission).  
    \item Tuesday Nov 29 or Thurs Dec 1: Presentation explaining your project topic to the class (group submission).  
    
    1 person team: 2 minutes.  2 person team: 3.5 minutes.  3 person team: 4.5 minutes.
    \item Due Friday Dec 2: Weekly project update and work log (individual submission).  
    
    Draft final project summary (group submission).
    \item Due Wednesday Dec 7: Peer review on final project summary (individual submission)
    \item Due Friday Dec 9: Weekly project update and work log (individual submission).
    \item Due Thursday Dec 15: Final project summary (group submission).  
    
    Individual project report (individual submission).
\end{itemize}


\end{document}